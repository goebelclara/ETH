\section{Syntax | Dependency Parsing}
\subsection*{Description}
\emph{Task} --- Assign syntactic structure, i.e., parse tree, to a sentence by breaking it down into a hierarchy of dependencies (similar to syntactic parsing)

{\color{black}\hrule height 0.001mm}

\subsection*{Formulation}
\begin{itemize}
    \item Done via \emph{spanning tree}:
    \begin{itemize}
        \item Connects $N$ nodes via $N-1$ edges
        \item Contains no cycles
    \end{itemize}
    \item For dependency parsing, we also require:
    \begin{itemize}
        \item The tree has exactly one root node, with only one outgoing edge
        \item All non-root nodes have exactly one incoming edge
        \item Edges are directed
        \item Edges are labeled
    \end{itemize}
    \item We can turn syntactic parsing into dependency parsing based on rules
    \item E.g. \hl{INSERT IMAGE 1}
    \item Types of dependency trees:
    \begin{itemize}
        \item \emph{Projective trees}:
        \begin{itemize}
            \item No crossing arcs
            \item Closely related to constituents (syntactic parsing)
            \item Can be treated with algorithms similar to CKY
            \item Note: In syntactic parsing, dependency relationships are always nested within constituents, meaning that syntactic parsing trees are always projective
            \item Not in focus in the following
        \end{itemize}
        \item \emph{Non-projective trees}:
        \begin{itemize}
            \item Crossing arcs
            \item In focus in the following
        \end{itemize}
    \end{itemize}
\end{itemize}

{\color{lightgrey}\hrule height 0.001mm}

\emph{Probability of spanning tree} ---
\begin{itemize}
    \item 
    $
    p(t \mid w) = \frac{1}{Z} \exp(\textrm{score}(t, w))
    $ 
    where 
    $Z = \sum_{t'} \exp(\textrm{score}(t', w))$
    \item Challenge: Naively, if $|w| = N$:
    \begin{itemize}
        \item It takes $\mathcal{O}(N^N)$ time to compute $Z$, if we don't impose a spanning tree structure and every node can depend on any other node, including itself
        \item There are $\mathcal{O}(N^{N-2})$ spanning trees we can form from $N$ nodes, if we impose a spanning tree structure constraint
        \item There are $\mathcal{O}((N-1)^{N-2})$ spanning trees we can form from $N$ nodes, if we impose the requirement that there is exactly one root node
    \end{itemize}
    \item Solution: 
    \begin{itemize}
        \item \emph{Edge factored assumption}: $
        p(t \mid w) = \frac{1}{Z} \prod_{(i \to j) \in t} \exp(\textrm{score}(i, j, w)) \exp(\textrm{score}(r, w))
        $ where $(i \to j)$ is an edge in the tree, $r$ is the root, and
        $
        Z = \sum_{t'} \prod_{(i \to j) \in t'} \exp(\textrm{score}(i, j, w)) \exp(\textrm{score}(r, w))
        $
        \item \emph{Kirchhoff's Matrix-Tree Theorem}: Method for counting the number of undirected spanning trees in $\mathcal{O}(N^3)$ time
        \item \emph{Tutte's Matrix-Tree Theorem}: Generalizes Kirchhoff's Matrix-Tree Theorem to directed spanning trees:
        \begin{itemize}
            \item Take adjacency matrix and root vector:
            \begin{itemize}
                \item \emph{Adjacency matrix}:
                $
                A_{ij} = \exp(\textrm{score}(i, j, w))
                $
                \item \emph{Root vector}:
                $
                \rho_j = \exp(\textrm{score}(j, w))
                $
            \end{itemize}
            \item Construct \emph{Laplacian matrix}:
            Not accounting for constraint that there is only one root node:
            $L_{ij} =
            \begin{cases}
                A_{ij} & \text{if } i \neq j \\
                \rho_j + \sum_{k \neq i} A_{kj} & \text{otherwise}
            \end{cases}$\\
            Accounting for constraint that there is only one root node:
            $L_{ij} =
            \begin{cases}
                \rho_j & \text{if } i = 1 \\
                \sum_{k=1,k \neq i} A_{kj} & \text{if } i = j \\
                -A_{ij} & \text{otherwise}
            \end{cases}$
            \item $|L| = Z$
        \end{itemize}
        \item Then, we can compute $Z$ under the edge factored assumption in $\mathcal{O}(N^3)$
        \item We optimize log-likelihood to derive parameters:
        $
        \sum_{i=1}^l \log p(t^{(i)} \mid w^{(i)}) = \sum_{i=1}^l \textrm{score}(t^{(i)}, w^{(i)}) - \log Z(w^{(i)})
        $ Gradient of log-likelihood can be computed via backpropagation in $\mathcal{O}(N^3)$
        \item 
    \end{itemize}
\end{itemize}

{\color{lightgrey}\hrule height 0.001mm}

\emph{Construct and decode tree} ---
Construct tree:
\begin{itemize}
    \item $N = |w|$ nodes plus 1 extra root node
    \item $(N+1)^{(N+1)}$ edges
    \item Each edge with a label and weight
\end{itemize}

Decode tree:
\begin{itemize}
    \item Task of finding the maximum-weight directed spanning tree
    \item Challenge: To perform decoding, greedy \emph{Kruskal's algorithm} does not work, since this selects the maximum-weight arc and then builds from there, which may be suboptimal in the directed case
    \item Solution: \emph{Chu-Liu/Edmonds algorithm}
\end{itemize}

Chu-Liu/Edmonds algorithm:
\hl{INSERT IMAGES}
\begin{itemize}
    \item Without root constraint: Can be run in $\mathcal{O}(N^2)$ time
    \begin{enumerate}
        \item Greedy algorithm selects best incoming edge for each node\\
        $\rightarrow$ this can cause cycles
        \item Contract cycles
        \begin{itemize}
            \item Edges fully in the cycle are \textcolor{gray}{dead}
            \item Edges fully outside the cycle are \textcolor{teal}{external}
            \item Edges exiting the cycle are \textcolor{blue}{exits}
            \item Edges entering  the cycle are \textcolor{magenta}{enters}
        \end{itemize}
        \item Break cycle: For each edge, break cycle by removing edges that are also incoming at the node where the enter edge is
        \item Reweight: Add weights of remaining edges to enter edge
        \item Now, greedy algorithm selects tree without cycles
        \item Then, we can re-expand: Pick edge of highest-scoring enter edge in contracted form, pick edges of remaining dead edges for that enter edge, which were used to re-weight
    \end{enumerate}
    \item With root constraint:
    \begin{itemize}
        \item Naively: Run above algorithm $N$ times, fixing each edge as the only one emanating from the root, adds factor $N$ to runtime
        \item Solution: Adjusted algorithm
    \end{itemize}
    \begin{enumerate}
        \item Run steps $1-4$ as above
        \item If there are multiple edges emanating from the root: For each root edge, calculate cost of deleting this edge: Cost = Weight of root edge - weight of next-best incoming edge to target node
        \item Remove edge with lowest cost
        \item Re-run greedy algorithm in contracted form
        \item If this leads to a cycle: Contract, then re-expand\\
        Otherwise: Re-expand
    \end{enumerate}
\end{itemize}