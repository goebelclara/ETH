\section{Weighted Finite State Automata (WFSA)}
\subsection*{Finite State Automata (FSA)}
\emph{Task} --- Determine whether a string is an element of a given language

{\color{lightgrey}\hrule height 0.001mm}

\emph{Formalization} --- 
\begin{itemize}
    \item $\mathcal{A}$ is a 5-tuple $(\Sigma, Q, I, F, \delta)$
    \begin{itemize}
        \item $\Sigma$ is an alphabet
        \item $Q$ is a finite set of states
        \item $I \subseteq Q$ is the set of initial states
        \item $F \subseteq Q$ is the set of final or accepting states
        \item $\delta \subseteq Q \times (\Sigma \cup \{\varepsilon\}) \times Q$ resp. $q \xrightarrow{a \in \Sigma \cup \{\varepsilon\}} q'$ is a finite multiset of transitions from one state in $Q$ to another state in $Q$ via a symbol in $\Sigma$ 
    \end{itemize}
    \item Can be represented as a directed, possibly cyclical graph:
    \hl{INSERT IMAGE 1}
    \item Sequentially reads individual symbols of an input string $s$ and transitions from state $q$ to state $q'$ upon reading a symbol $a$ iff $(q, a, q') \in \delta$
    \item If the automaton, after reading the last symbol of $s$, ends up in a state $q_f \in F$, the automaton accepts the string
\end{itemize}

{\color{black}\hrule height 0.001mm}

\subsection*{Weighted Finite State Automata (WFSA)}
\begin{itemize}
    \item Generalization of FSA, where transitions are weighted with a semiring
    \item Unweighted FSA corresponds to WFSA weighted with Boolean semiring
    \item Formalization: $\mathcal{A}$ is a 7-tuple $(\Sigma, Q, I, F, \delta, \lambda, \rho)$ over a semiring $\mathcal{W} = (\mathbb{K}, \oplus, \otimes, 0, 1)$.
    \begin{itemize}
        \item $\Sigma$ is an alphabet
        \item $Q$ is a finite set of states
        \item $I = \{q \in Q \mid \lambda(q) \neq 0\} \subseteq Q$ is the set of initial states
        \item $F = \{q \in Q \mid \rho(q) \neq 0\} \subseteq Q$ is the set of final or accepting states
        \item $\delta \subseteq Q \times (\Sigma \cup \{\varepsilon\}) \times \mathbb{K} \times Q$ is a finite multiset of transitions from one state in $Q$ to another state in $Q$ via a symbol in $\Sigma$
        \item $\lambda : Q \to \mathbb{K}$ is an initial weighting function over $Q$
        \item $\rho : Q \to \mathbb{K}$ is a final weighting function over $Q$
    \end{itemize}
    \item Can be represented as a directed, possibly cyclical graph:
    \hl{INSERT IMAGE 2}
\end{itemize}

{\color{black}\hrule height 0.001mm}

\subsection*{(W)FSA Terminology}
\begin{itemize}
    \item Paths:
    \begin{itemize}
        \item A path $\pi$ is an element of $\delta^*$ with consecutive transitions:
        $
        q_1 \xrightarrow{a_1} q_2 \xrightarrow{a_2} q_3 \cdots q_{N-1} \xrightarrow{a_N} q_N
        $
        \item $\textrm{p}(\pi) = q_1$ is the beginning state of the path
        \item $\textrm{q}(\pi) = q_N$ is the ending state of the path
        \item Length of path $|\pi|$ is the number of transitions
        \item Yield of path $s(\pi)$ is the concatenation of symbols on the path
        \item Path sets: Denoted by capital $\Pi$
        \begin{itemize}
            \item $\Pi(\mathcal{A})$: Set of all paths in the automaton
            \item $\Pi(s, \mathcal{A})$: Set of all paths with yield $s$
            \item $\Pi(q, \mathcal{A})$: Set of all paths starting at $q$
            \item $\Pi(q, q', \mathcal{A})$: Set of all paths from $q$ to $q'$
            \item $\Pi(q, s, q', \mathcal{A})$: Set of all paths from $q$ to $q'$ with yield $s$
            \item $\Pi(\mathcal{Q}, \mathcal{A}) = \bigcup_{q \in Q} \Pi(q)$
            \item $\Pi(\mathcal{Q}, \mathcal{Q}', \mathcal{A}) = \bigcup_{q \in Q, q' \in Q'} \Pi(q, q')$
            \item $\Pi(\mathcal{Q}, s, \mathcal{Q}' \mathcal{A}) = \bigcup_{q \in Q, q' \in Q'} \Pi(q,s,q')$
        \end{itemize}
    \end{itemize}

    \item Cycles:
    \begin{itemize}
        \item A path is a cycle if starting and finishing states are the same
        \item A path contains a cycle if the same state appears multiple times
    \end{itemize}

    \item Transitions:
    \begin{itemize}
        \item Outgoing arcs from $q$: $E \mathcal{A} (q) = \{a, b, \ldots\ \mid (q, a, b, \ldots) \in \delta\}$
        \item Incoming arcs to $q$: $E^{-1} \mathcal{A} (q) = \{c, d, \ldots\ \mid (\ldots, c, d, q) \in \delta\}$
    \end{itemize}

    \item States:
    \begin{itemize}
        \item State $q$ is accessible iff $q \in I$ or there exists a path whose weight is not 0 from $I$ to $q$
        \item State $q$ is accessible iff $q \in F$ or there exists a path whose weight is not 0 from $q$ to $F$
    \end{itemize}

    \item (W)FSA is unambiguous iff for every string $s$ there is at most one accepting path
\end{itemize}

{\color{black}\hrule height 0.001mm}

\subsection*{(W)FSA Applications}
N-gram models, CRFs, HMMs can be represented with (W)FSAs

\emph{N-gram models} --- 
\begin{itemize}
    \item Let $\boldsymbol{y} = (y_1, \ldots, y_L)$ be a string with $y_1 = \textrm{BOS}$, $y_L = \textrm{EOS}$
    \item We want to model 
    $
    p(\boldsymbol{y}) = \prod_{n=1}^L p(y_n \mid y_{n-N+1}, \ldots, y_{n-1})
    $
    \item If we represent this with WFSA:
    \begin{itemize}
        \item States $Q$ represent all possible sequences of words of length $N$, i.e., all possible N-grams
        \item There are $m + |V|^{n-1} + 1$ states for an N-gram:
        \begin{itemize}
            \item 1 end state
            \item $|V|^{n-1}$ intermediate states
            \item $m = \begin{cases} 
                |V|^{n-2} & \textrm{for N-grams with } n \geq 2 \\ 
                0 & \textrm{for unigrams} 
            \end{cases}$
        \end{itemize}
        \item Transitions between states represent transitions from one N-gram to another \item Only transitions between N-grams which can follow each other are allowed, e.g.
        $
        y_{-N} \ y_{-N+1} \dots y_{-1}
        $
        and 
        $
        y_{-N+1} \ y_{-N+2} \dots y_{0}
        $
        receive weights $> 0$, with the weight representing the probability of observing the new word $y_0$ given the starting N-gram
    \end{itemize}
    \item Formalization:
    \begin{itemize}
        \item $\Sigma = \mathbb{Y} \cup \{\langle \textrm{BOS} \rangle, \langle\textrm{EOS} \rangle\}$
        \item $Q = \bigcup_{n=0}^N \{\langle \textrm{BOS} \rangle\}^{N-n} \times \mathbb{Y}^{n-1} \times (\mathbb{Y} \cup \{\langle \textrm{EOS} \rangle\})$
        \item $I = \{\langle \textrm{BOS}, \ldots, \textrm{BOS} \rangle\}$ ($N$ times)
        \item $F = Q$
        \item $\delta = \big\{ (y_{-N} \ldots y_{-1}, y_0, p(y_0 \mid y_{-N+1}, \ldots, y_{-1}), y_{-N+1} \ldots y_0 ) \textrm{ for } (y_{-N+1} \ldots y_{-1}) \in \bigcup_{n=0}^{N-1} \{ \langle \textrm{BOS} \rangle \}^{N-1-n} \times \mathbb{Y}^n  \big\}
        $
        \\
        $
        \delta(q_{ij}, w, 1, q_{ki}) = 
        \begin{cases} 
            \log(p(w_m = k \mid w_{m-1} ) = i, \ldots, w_{m-n+1} = j)) & \textrm{if } w=k \\
            -\infty & \textrm{otherwise}
        \end{cases}
        $
        \item $\lambda = \langle \textrm{BOS} \rangle \ldots \langle \textrm{BOS} \rangle \to 1$ ($N$ times)\\
        $\lambda(q_{ij}) = \log(p(w_m = l \mid w_{m-1} = \textrm{BOS}, \ldots, w_{m-n+1} = \textrm{BOS}))$
        \item $\rho = y_{n-N+1} \ldots y_1 \ \langle \textrm{EOS} \rangle \to 1$\\
        $
        \rho(q_{ij}) = \log(p(w_m = \textrm{EOS} \mid w_{m-n} = i, \ldots, w_{m-n+1} = j))
        $
    \end{itemize}
\end{itemize}

{\color{lightgrey}\hrule height 0.001mm}

\emph{CRFs} --- 
\begin{itemize}
    \item Let $|\boldsymbol{x}| = L$ and $|\boldsymbol{y}| = M$
    $
    \textrm{score}(\boldsymbol{y}, \boldsymbol{x}) = \boldsymbol{w} \cdot \boldsymbol{f}(\boldsymbol{x}, \boldsymbol{y})
    $
    \item We assume $\boldsymbol{f}$ decomposes additively:
    $
    \boldsymbol{f}(\boldsymbol{x}, \boldsymbol{y}) = \sum_{n=2}^L f(y_n, y_{n-1}, \boldsymbol{x}, n)
    $
    \item We want to model:
    $
    p(\boldsymbol{y} \mid \boldsymbol{x}) = \frac{\exp(\textrm{score}(\boldsymbol{y}, \boldsymbol{x}))}{Z(\boldsymbol{x})} = \frac{\prod_{n=2}^L \exp(\boldsymbol{w} \cdot f(y_n, y_{n-1}, \boldsymbol{x}, n))}{Z(\boldsymbol{x})}
    $
    where
    $
    Z(\boldsymbol{x}) = \sum_{\boldsymbol{y}'} \exp(\textrm{score}(\boldsymbol{y}', \boldsymbol{x})) = \sum_{\boldsymbol{y}'} \prod_{n=2}^L \exp(\boldsymbol{w} \cdot f(y_n', y_{n-1}', \boldsymbol{x}, n))
    $
    \item $Z(\boldsymbol{x})$ can be computed via the pathsum algorithm, as a generalization of the backward algorithm
    \item If we represent this with WFSA:
    \begin{itemize}
        \item States $Q$ represent all hidden states
        \item Transitions between states represent CRF scores assigned to hidden state $y'$ after hidden state $y$, given input sequence $\boldsymbol{x}$
    \end{itemize}
    \item Formalization:
    \begin{itemize}
        \item $\Sigma = \{\varepsilon\}$
        \item $Q = \mathbb{Y}^M$
        \item $I = Q$
        \item $F = Q$
        \item $\delta = \{(y_n, \varepsilon, \exp(\boldsymbol{w} \cdot f(y_n, y_{m}, \boldsymbol{x})), y_m) \}$
        \item $\lambda = y_n \to \exp(\boldsymbol{w} \cdot f(y_n, \langle \textrm{BOS} \rangle, \boldsymbol{x}))$
        \item $\rho = y_n \to \exp(\boldsymbol{w} \cdot f(\langle \textrm{EOS} \rangle, y_n, \boldsymbol{x}))$
    \end{itemize}
\end{itemize}

\emph{HMMs} --- 
\hl{TBA}

{\color{black}\hrule height 0.001mm}

\subsection*{Weighted Finite State Transducers (WFST)}
\emph{Task} --- Transforms input string in a given language to output string in a given language

{\color{lightgrey}\hrule height 0.001mm}

\emph{Formalization} --- 
\begin{itemize}
    \item $\mathcal{T}$ is an 8-tuple $(\Sigma, \Omega, Q, I, F, \delta, \lambda, \rho)$ over a semiring $\mathcal{W} = (\mathbb{K}, \oplus, \otimes, 0, 1)$
    \begin{itemize}
        \item $\Sigma$ is the input alphabet, containing letters or symbols
        \item $\Omega$ is the output alphabet, containing letters or symbols
        \item $Q$ is a finite set of states
        \item $I = \{q \in Q \mid \lambda(q) \neq 0\} \subseteq Q$ is the set of initial states
        \item $F = \{q \in Q \mid \rho(q) \neq 0\} \subseteq Q$ is the set of final or accepting states
        \item $\delta \subseteq Q \times (\Sigma \cup \{\varepsilon\}) \times (\Omega \cup \{\varepsilon\}) \times \mathbb{K} \times Q$ is a finite multiset of transitions from one state in $Q$ to another state in $Q$ via a symbol in $\Sigma$ and in $\Omega$
        \item $\lambda : Q \to \mathbb{K}$ is an initial weighting function over $Q$
        \item $\rho : Q \to \mathbb{K}$ is a final weighting function over $Q$
    \end{itemize}
    \item Furthermore:
    \begin{itemize}
        \item $s$: String over alphabet
        \item $\varepsilon$: Empty string
        \item $\Sigma^0 = \{\varepsilon\}$
        \item $\Sigma^* = \{\varepsilon, \ldots\}$ is the set of all possible strings (incl. the empty string $\varepsilon$) that can be formed by concatenation of 0 or more elements from $\Sigma$ 
        \item $L$: String language over an alphabet, subset of $\Sigma^*$, containing words
        \item $\boldsymbol{x}$: Input string
        \item $\boldsymbol{y}$: Transliteration of input string
    \end{itemize}
    \item Can be represented as a directed, possibly cyclical graph
    \hl{INSERT IMAGE 3}
\end{itemize}

{\color{black}\hrule height 0.001mm}

\subsection*{Compositions}
\begin{itemize}
    \item Operation to combine two or more WFSTs
    \item $\mathcal{T}_1 \circ \mathcal{T}_2$ is the composition of $\mathcal{T}_1 = (\Sigma, \Omega, Q_1, I_1, F_1, \delta_1, \lambda_1, \rho_1)$ and $\mathcal{T}_2 = (\Omega, \Gamma, Q_2, I_2, F_2, \delta_2, \lambda_2, \rho_2)$ and is given by:
    $
    \mathcal{T} = (\Sigma, \Gamma, Q, I, F, \delta, \lambda, \rho)
    $
    such that:
    $
    \mathcal{T}(x, y) = \bigoplus_{z \in \Omega^*} \mathcal{T}_1(x, z) \otimes \mathcal{T}_2(z, y)
    $
    \item Naive algorithm to compute composition:
    \begin{enumerate}
        \item $\mathcal{T} \gets (\Sigma, \Omega, Q, I, F, \delta, \lambda, \rho)$ to create a new WFST
        \item For $q_1, q_2 \in Q_1 \times Q_2$:\\
        For
        $
        (q_1 \xrightarrow{a:b/w_1} q_1'), (q_2 \xrightarrow{c:d/w_2} q_2') \in  E_{\mathcal{T}_1}(q_1) \times E_{\mathcal{T}_2}(q_2):
        $\\
        If $b=c$
        \begin{itemize}
            \item Add states $(q_1, q_2)$, $(q_1', q_2')$ to $\mathcal{T}$ if they have not yet been added
            \item Add arc $(q_1, q_2) \xrightarrow{a:d/w_1 \otimes w_2} (q_1', q_2')$ to $\mathcal{T}$
        \end{itemize}
        \item For $(q_1, q_2) \in Q$:
        \begin{itemize}
            \item $
            \lambda_\mathcal{T} = \lambda_1(q_1) \otimes \lambda_2(q_2)
            $
            \item $
            \rho_\mathcal{T} = \rho_1(q_1) \otimes \rho_2(q_2)
            $
        \end{itemize}
        \item Return $\mathcal{T}$
    \end{enumerate}
    \item Challenge: Runs through many useless states, has time complexity $\mathcal{O}(|Q_1||Q_2|)$
    \item Solution: Accessible algorithm
    \item Accessible algorithm:
    \begin{itemize}
        \item Intuition: Construct all possible pairs of initial states and then expand outwards (depth-first search), adding accessible states
        \item Attention: Produced states may still be non-co-accessible
        \begin{enumerate}
            \item $\textrm{stack} \gets [(q_1, q_2) \mid q_1 \in I_1, \, q_2 \in I_2]$ (ordered list, allows duplicates)\\
            $\textrm{visited} \gets \{(q_1, q_2) \mid q_1 \in I_1, \, q_2 \in I_2\}$ (unordered set, does not allow duplicates)
            \item $\mathcal{T} \gets (\Sigma, \Omega, Q, I, F, \delta, \lambda, \rho)$ to create a new WFST
            \item While $|\textrm{stack}| > 0$:
            \begin{itemize}
                \item $q_1, q_2 \gets \textrm{stack.pop()}$
                \item For:
                $
                (q_1 \xrightarrow{a:b/w_1} q_1'), \quad (q_2 \xrightarrow{c:d/w_2} q_2') \in E_{\mathcal{T}_1}(q_1) \times E_{\mathcal{T}_2}(q_2)$:\\
                If $b=c$:
                \begin{itemize}
                    \item Add states $(q_1, q_2)$, $(q_1', q_2')$ to $\mathcal{T}$ if they have not yet been added
                    \item Add arc $(q_1, q_2) \xrightarrow{a:d/w_1 \otimes w_2} (q_1', q_2')$ to $\mathcal{T}$
                    \item If $(q_1', q_2')$ not in visited:
                    \begin{itemize}
                        \item Push $(q_1', q_2')$ to stack and visited
                    \end{itemize}
                \end{itemize}
            \end{itemize}
            \item For $(q_1, q_2) \in Q$:
            \begin{itemize}
                \item $
                \lambda_\mathcal{T} = \lambda_1(q_1) \otimes \lambda_2(q_2)
                $
                \item $
                \rho_\mathcal{T} = \rho_1(q_1) \otimes \rho_2(q_2)
                $
            \end{itemize}
        \end{enumerate}
    \end{itemize}
    \item E.g.
    \hl{INSERT IMAGE 4}
\end{itemize}

{\color{black}\hrule height 0.001mm}

\subsection*{Transliteration}
\emph{Basics} ---
\begin{itemize}
    \item Mapping strings in one character set to strings in another character set E.g. $\textrm{Muse} \to \mu ou \sigma \alpha$
    \item Approach:
    \begin{enumerate}
        \item Design WFST $\mathcal{T}$ that maps string over $\Sigma$ to string over $\Omega$: $\mathcal{T}$ acts as a rule set
        \item Encode source sequence $\boldsymbol{x} \in \Sigma^*$ as WFST $\mathcal{T}_x$: Prepare $\boldsymbol{x}$ for alignment
        \item Compute normalizer $Z$ as pathsum of composition $\mathcal{T}_x \circ \mathcal{T}_y$: Compute denominator of $p(\boldsymbol{y} \mid \boldsymbol{x})$
        \item Encode target sequence $\boldsymbol{y} \in \Omega^*$ as WFST $\mathcal{T}_y$: Prepare $\boldsymbol{y}$ for alignment
        \item Sum over all possible alignments using pathsum of composition $\mathcal{T}_x \circ \mathcal{T} \circ \mathcal{T}_y$: Compute numerator of $p(\boldsymbol{y} \mid \boldsymbol{x})$
        \item Compute highest scoring path of composition $\mathcal{T}_x \circ \mathcal{T}$ using e.g. Dijkstra or Floyd-Warshall: Identify most probable transliteration
    \end{enumerate}
    \item How do we design $\mathcal{T}$? Path $\pi$ from $\mathcal{T}$'s initial states to a final state represents an alignment of $\boldsymbol{x}$ to $\boldsymbol{y}$
    \item How do we use training data?
    \begin{itemize}
        \item Dataset contains pairs $(\boldsymbol{x}, \boldsymbol{y})$, where sometimes there is an alignment between $\boldsymbol{x}$ and $\boldsymbol{y}$
        \item We treat the alignment as a latent variable
        \item If we represent $\boldsymbol{x}$ as transducer $\mathcal{T}_x$, the composition $\mathcal{T}_x \circ \mathcal{T}$ yields a new transducer whose paths correspond to the paths in $\mathcal{T}$ that have input $\boldsymbol{x}$: Each path in $\mathcal{T}_x \circ \mathcal{T}$ represents one alignment of $\boldsymbol{x}$ to some string in $\Omega^*$
        \item If we represent $\boldsymbol{y}$ as transducer $\mathcal{T}_y$, the composition $\mathcal{T}_x \circ \mathcal{T} \circ \mathcal{T}_y$ yields a new transducer whose paths correspond to the paths in $\mathcal{T}$ that align $\boldsymbol{x}$ to $\boldsymbol{y}$
        \item The probability $p(\boldsymbol{y} \mid \boldsymbol{x})$ is the total probability of all paths that align $\boldsymbol{x}$ to $\boldsymbol{y}$
    \end{itemize}
    \item How do we train? Maximize log-likelihood 
    $\sum_{\boldsymbol{x}, \boldsymbol{y} \in \mathcal{D}} \log p(\boldsymbol{y} \mid \boldsymbol{x}) = \textrm{pathsum of } \mathcal{T}_x \circ \mathcal{T} \circ \mathcal{T}_y, \textrm{ divided by pathsum of } \mathcal{T}_x \circ \mathcal{T}$
    \item How do we perform inference?
    \begin{itemize}
        \item Given $\boldsymbol{x}$
        \item Construct $\mathcal{T}_x \circ \mathcal{T}$, defining a probability distribution over all aligned $\boldsymbol{y}$
        \item If we condition on $\boldsymbol{x}$ in WFST $\mathcal{T}_x \circ \mathcal{T}$, this WFST becomes a WFSA
        \item Goal is then to find the highest scoring path in this graph
    \end{itemize}
    \item Why can't we use a CRF?
    \begin{itemize}
        \item In CRFs, we can only align sequences of the same length, e.g., $|\boldsymbol{x}| = |\boldsymbol{y}|$
        \item In WFSTs, we can align sequences of different lengths
    \end{itemize}
\end{itemize}

{\color{lightgrey}\hrule height 0.001mm}

\emph{Pathsum} ---
\begin{itemize}
    \item Inner path weight $w_\mid(\pi)$ of a path $\pi = q_0 \xrightarrow{a_1/w_1} q_1 \cdots q_{N-1} \xrightarrow{a_N/w_N} q_N$:
    $
    w_\mid(\pi) = \bigotimes_{n=1}^N w_n
    $
    \item Path weight $w(\pi)$ of a path:
    $
    w(\pi) = \lambda(\textrm{p}(\pi)) \otimes w_\mid \otimes \rho(\textrm{q}(\pi))
    $
    where $\textrm{p}(\pi)$ is the beginning state and $\textrm{q}(\pi)$ is the ending state
    \item A path is accepting or successful iff $w(\pi) \neq 0$
    \item String sum of a string $s \in \Sigma^*$ under $\mathcal{A}$:
    $
    A(s) = \bigoplus_{\pi \in \Pi(I, s, F)} w(\pi)
    $
    \item Path sum of $\mathcal{A}$:
    $
    Z(\mathcal{A}) = \bigoplus_{\pi \in \Pi(\mathcal{A})} w(\pi)
    $
    \item Path sum between two states $q_n, q_m$:
    $
    Z(q_n, q_m) = \bigoplus_{\pi \in \Pi(q_n, q_m)} w(\pi)
    $
    \item Path sum is equal to the shortest path weight, which can be seen, e.g., with the tropical semiring where $\otimes = +$ and $\oplus = \min$
    \item In a $\overline{0}$-closed semiring, the shortest path weight depends only on paths of length $N-1$, since paths of length $\geq N$ contain cycles, but cycles do not improve the path sum in $\overline{0}$-closed semirings, since $\overline{1} + a = \overline{1}$
    \item Challenge: There may be an infinite number of accepting paths (guaranteed if WFSA is cyclical)
    \item Nonetheless, we can compute the pathsum
    \item The pathsum problem subsumes several other problems:
    \begin{itemize}
        \item N-gram normalization
        \item CRF normalization
        \item Computing marginals in HMMs
    \end{itemize}
    \item Pathsum in acyclic WFSA:
    \begin{itemize}
        \item In acyclic WFSA, the number of accepting paths is finite
        \item To compute pathsum $Z(\mathcal{A})$, we need a topological sort of the nodes
        \item Backward algorithm:
        \begin{itemize}
            \item Computes $Z(\mathcal{A})$
            \begin{enumerate}
                \item For $q \in \textrm{Rev-Top}(\mathcal{A})$: (sorts nodes topologically)
                $
                \beta(q) = \rho(q) \oplus \bigoplus_{q \xrightarrow{a/w} q'} w \otimes \beta(q')
                $ i.e. proceeds in reverse topological order, $\oplus$-summing all weights of all paths coming from node $q$ and ending in final state $q_F$.
                \item Return:
                $
                \bigoplus_{q \in I} \lambda(q_\mid) \otimes \beta(q_\mid)
                $
            \end{enumerate}
            \item Runs in $\mathcal{O}(|\delta|)$ time and $\mathcal{O}(|Q|)$ space
            \item Could alternatively also be computed as forward algorithm. Then we would:
            \begin{itemize}
                \item Visit incoming edges (instead of outgoing)
                \item $\oplus$-add the initial weights (instead of final weights) in step (1)
                \item Post-$\otimes$-multiply the final weights (instead of initial weights) in step (2)
            \end{itemize}
        \end{itemize}
    \end{itemize}
    \item Pathsum in cyclical WFSA:
    \begin{itemize}
        \item In cyclical WFSA, the number of accepting paths is infinite
        \item To compute pathsum $Z(\mathcal{A})$, we need a closed semiring
        \item Lehmann algorithm:
        \begin{itemize}
            \item Computes $Z(\mathcal{A})$ by computing Kleene closure of matrixes over arbitraty closed semirings
            \item We must first represent WFSA as a matrix:
            \begin{itemize}
                \item Define $|\Sigma|$ \emph{adjacency matrices}, one for each transition symbol: $\boldsymbol{W}(a)$ is the adjacency matrix of transitions over $a$
                \item \hl{INSERT IMAGE 5}
                \item Since all paths are summed, we can collapse all transitions from $q_n$ to $q_m$ into a single transition without a label, whose weight is the $\oplus$-sum of all original transitions:
                $
                W(\mathcal{A}) = \bigoplus_{a \in \Sigma \cup \{\varepsilon\}} \boldsymbol{W}(a)
                $
            \end{itemize}
            \item Broad steps:
            \begin{enumerate}
                \item Lehmann algorithm computes the $\oplus$-sum over the path between two nodes $q_i, q_k$ with the matrix $\boldsymbol{W}$:
                $
                \boldsymbol{R}_{ik} = \bigoplus_{\pi \in \Pi(q_i, q_k)} w_\mid(\pi)
                $
                \begin{itemize}
                    \item Cyclical terms in calculation of $\boldsymbol{R}_{ik}$ are denoted with $*$
                    \item $\boldsymbol{R}_{ik}$ has $|Q|^2$ entries
                \end{itemize}
                \item From $\boldsymbol{R}$, we can compute $Z(\mathcal{A})$:
                $
                Z(\mathcal{A}) = \bigoplus_{i,k=1}^{|Q|} \lambda(q_i) \otimes \boldsymbol{R}_{ik} \otimes \rho(q_k)
                $
            \end{enumerate}
            \item E.g.
            \hl{INSERT IMAGE 6}
            \item Intuition:
            \begin{itemize}
                \item Partition path $\pi$ between $q_i$ and $q_k$ into subsets of paths:
                $\boldsymbol{M}^{\leq j}(q_i, q_k)$ are paths that do not cross nodes with indices $> j$\\
                $\boldsymbol{R}_{ik}^{\leq j}$ is the pathsum for these paths
                \item Any path either crosses $j$ or not
                \item For paths that do not cross $j$: 
                \begin{itemize}
                    \item $
                    \pi \in \Pi^{\leq j-1}(q_i, q_k)
                    $
                    \item In that case, the pathsum is:
                    $
                    \boldsymbol{R}_{ik}^{\leq j-1}
                    $
                \end{itemize}
                \item For paths that do cross $j$: 
                \begin{itemize}
                    \item $\pi$ can be decomposed into cycle that starts and ends in $j$, part before cycle, and part after cycle: $
                    \pi = \pi_{ij} \pi_{jj} \pi_{jk}
                    $
                    \item In that case, we can also decompose the pathsum into partial pathsums:
                    $
                    \boldsymbol{R}_{ik}^{\leq j} = \boldsymbol{R}_{ij}^{\leq j-1} \otimes \boldsymbol{R}_{jj}^{\leq j-1 *} \otimes \boldsymbol{R}_{jk}^{\leq j-1}
                    $
                \end{itemize}
                \item This defines a natural procedure to build up the pathsum from the partial pathsums
                \item All that's left is the starting condition:
                $
                R^{\leq 0} = \boldsymbol{W}
                $ since $\boldsymbol{W}$ captures paths not passing through intermediary nodes
            \end{itemize}
            \item Algorithm:
            \begin{enumerate}
                \item $\boldsymbol{R}^{(0)} \gets \boldsymbol{W}$
                \item For $j \gets 1$ up to $|Q|$:\\
                For $i \gets 1$ up to $|Q|$:\\
                For $k \gets 1$ up to $|Q|$:
                $
                \boldsymbol{R}_{ik}^{(j)} \gets \boldsymbol{R}_{ik}^{(j-1)} \oplus \big(\boldsymbol{R}_{ij}^{(j-1)} \otimes \boldsymbol{R}_{jj}^{(j-1)*} \otimes \boldsymbol{R}_{jk}^{(j-1)}\big)
                $
                \item Return:
                $
                I \bigoplus \boldsymbol{R}^{(|Q|)}
                $
            \end{enumerate}
            \item E.g.
            \hl{INSERT IMAGE 7}
            \item Runs in $\mathcal{O}(|Q|^3)$ time
            \item Lehmann algorithm encompasses Floyd-Warshall algorithm and Gauss-Jordan algorithm
        \end{itemize}
    \end{itemize}
\end{itemize}

{\color{lightgrey}\hrule height 0.001mm}

\emph{Transliteration} ---
Pathsum in $\boldsymbol{x}$-conditioned WFSA:    \begin{itemize}
    \item Let $(\boldsymbol{x}, \boldsymbol{y})$ be an unaligned training pair
    \item $\mathcal{A}_x$ is the $\boldsymbol{x}$-conditioned WFSA:
    \begin{itemize}
        \item $\boldsymbol{\lambda}$ is a vector of start weights for starting states: $\lambda_n = \lambda(q_n)$
        \item $\boldsymbol{\rho}$ is a vector of end weights for ending states: $\rho_m = \rho(q_m)$
        \item Set of transition matrices $\boldsymbol{W}^{(\omega)}$ for any $\omega \in \Omega \cup \{\varepsilon\}$:
        $
        \boldsymbol{W}_{nm}^{(\omega)}$ is the weight for $q_n \xrightarrow{\omega} q_m
        $
        \item $\boldsymbol{W}^{(\omega)}$ can be summarized into a single matrix:
        $
        \boldsymbol{W} = \sum_{\omega \in \Omega \cup \{\varepsilon\}} \boldsymbol{W}^{(\omega)}
        $
    \end{itemize}
    \item For instance: Likelihood:
    $
    p(\boldsymbol{y} \mid \boldsymbol{x}) = \frac{\exp(\textrm{score}(\boldsymbol{x}, \boldsymbol{y}))}{Z(\boldsymbol{x})}
    $ 
    \item
    $
    \textrm{score}(\boldsymbol{x}, \boldsymbol{y}) = \log \sum_{\pi \in \Pi(\boldsymbol{y})} w(\pi)
    $
    \item Normalizing constant is the pathsum:
    $
    Z(\boldsymbol{x}) = \sum_{\boldsymbol{y}' \in \Omega^*} \exp(\textrm{score}(\boldsymbol{x}, \boldsymbol{y}'))
    $\\
    $
    = \sum_{\boldsymbol{y}' \in \Omega^*} \exp \Big( \log \sum_{\pi \in \Pi(\boldsymbol{y}')} w(\pi) \Big)
    $\\
    $
    = \sum_{\boldsymbol{y}' \in \Sigma^*} \sum_{\pi \in \Pi(\boldsymbol{y}')} \lambda(\textrm{p}(\pi)) \otimes w_\mid(\pi) \otimes \rho(\textrm{q}(\pi))
    $\\
    $
    = \sum_{\boldsymbol{y}' \in \Sigma^*} \sum_{\pi \in \Pi(\boldsymbol{y}')} \lambda(\textrm{p}(\pi)) \otimes \prod_{n=1}^{| \pi |} w_n(\pi) \otimes \rho(\textrm{q}(\pi))
    $ which is the pathsum\\
    $
    = \sum_{q_n, q_m \in Q} \Big( \sum_{\pi \in \Pi(q_n, q_m)} w(\pi) \Big)
    $\\
    $
    = \sum_{q_n, q_m \in Q} \sum_{\pi \in \Pi(q_n, q_m)} \lambda(\textrm{p}(\pi)) \otimes w_\mid(\pi) \otimes \rho(\textrm{q}(\pi))
    $\\
    $
    = \sum_{q_n, q_m \in Q} \sum_{\pi \in \Pi(q_n, q_m)} \lambda(q_n) \otimes w_\mid(\pi) \otimes \rho(q_m)
    $\\
    $
    = \sum_{q_n, q_m \in Q} \lambda(q_n) \otimes (\sum_{\pi \in \Pi(q_n, q_m)} w_\mid(\pi)) \otimes \rho(q_m)
    $\\
    $
    = \sum_{q_n, q_m \in Q} \lambda(q_n) \otimes (\boldsymbol{W}^*)_{nm} \otimes \rho(q_m)
    $\\
    $
    = \boldsymbol{\lambda}^\intercal \boldsymbol{W}^* \boldsymbol{\rho}
    $ where $\boldsymbol{W}$ can be computed via Lehmann
    \item For instance: Log-likelihood: $\log p(\boldsymbol{y} \mid \boldsymbol{x})$
    \item For a full training dataset: Log-likelihood: $\sum_{n=1}^N \log p(\boldsymbol{y}^{(n)} \mid \boldsymbol{x}^{(n)}) = \sum_{n=1}^N \Big[ \textrm{score}(\boldsymbol{y}^{(n)}, \boldsymbol{x}^{(n)}) - \log Z(\boldsymbol{x}^{(n)}) \Big]$
    \item This can be optimized with backpropagation (in $\mathcal{O}(|Q|^3)$) and gradient descent
    \item To train transliteration WFST, we use the log semiring
    \item To estimate the probability of the most likely transliteration $\boldsymbol{y}$ for a given input $\boldsymbol{x}$, we use the arctic semiring
\end{itemize}