\section{Probability and Statistics}
\subsection*{Terminology}
\emph{Kolmogorov axioms} --- 
Probability space defined by:
\begin{itemize}
    \item Sample space: All possible outcomes $\Omega = \{\omega_1,...,\omega_n\}$
    \item Event space: All possible results, where an event is a subset of the sample space 
    \item Probability measure: Function that assigns a probability to an event 
\end{itemize}
Axioms:
\begin{itemize}
    \item Event space must be a \emph{sigma algebra}:
    \begin{itemize}
        \item If $A$ is in sample space with $P = a$, its complement is also in sample space with $P = 1-a$
        \item If $A_1,...A_n$ are in sample space with $P=a_1, ..., a_n$, their union is also in sample space with $P = a_1 + ... + a_n$
    \end{itemize}
    \item Probability measure must satisfy:
    \begin{itemize}
        \item $0 \leq \mathbb{P}(A) \leq 1$
        \item $\mathbb{P}({\Omega}) = 1$
        \item If $A_1,A_2,...$ are in sample space and do not intersect, then $\mathbb{P}(A_1 \cup A_2 \cup ...) = \int_{n=1}^{\infty} \mathbb{P}(A_n)$
    \end{itemize}
\end{itemize}
Further properties:
\begin{itemize}
    \item All sets than can be formed from left and right inclusive interval $[0,a]$ are events. On that basis:
    \begin{itemize}
        \item $(a,1] = [0,a]^c \in$ event space, with $P = 1 - a$
        \item $(a,b] = ([0,a] \cup (b,1])^c = ([0,a] \cup [0,b]^c)^c \in$ event space, with $P = 1 - (a + 1 - b) = 1 - a - 1 + b = b - a$
        \item $\{0\} \in$ event space, with $P = 0$
        \item $\{a\} \in$ event space, with $P = 0$
        \item $[a,b] = \{a\} \cup (a,b] = \{a\} \cup ([0,a] \cup [0,b]^c)^c \in$ event space, with $P = 0 + b - a$
        \item $[a,b) = [a,b] \backslash \{b\} = ([0,a] \cup [0,b]^c)^c \backslash \{b\} \in$ event space, with $P = b - a - 0$
    \end{itemize}
\end{itemize}

{\color{lightgray}\hrule height 0.001mm}

\emph{Variables} --- 
\begin{itemize}
    \item Random variable:
    \begin{itemize}
        \item Discrete random variable: Characterized by pmf
        \item Continuous random variable: Characterized by pdf
    \end{itemize}
    \item Independent random variables:
    \begin{itemize}
        \item $\mathbb{P}(A|B) = \mathbb{P}(A)$ and $\mathbb{P}(B|A) = \mathbb{P}(B)$
        \item From this follows that $f_{A|B}(a|b) = f_A(a)$ and $\mathbb{E}(A|B) = \mathbb{E}(A)$ 
        \item $\mathbb{P}(A \cap B) = \mathbb{P}(A)\mathbb{P}(B)$
        \item From this it follows that $f_{A,B}(a,b) = f_A(a) f_B(b)$ and $\mathbb{E}(A,B) = \mathbb{E}(A)\mathbb{E}(B)$
        \item Correlation is 0 
        \item Functions of independent random variables are also independent
    \end{itemize}
    \item Conditionally independent random variables: Two random variables $\mathcal{X}$ and $\mathcal{Y}$ are conditionally independent, if there is a confounder $\mathcal{L}$ that causally affects both variables, but if we control for this confounder, the variables are not causally connected
    \item I.I.D. random variables: Independent and from identical distribution
\end{itemize}

{\color{lightgray}\hrule height 0.001mm}

\emph{Events} --- 
\begin{itemize}
    \item Complement: $\mathbb{P}(A^C) = 1 - \mathbb{P}(A)$ and $\mathbb{P}(A \cup A^C) = \mathbb{P}(A)\mathbb{P}(A^C)$
    \item Disjoint / mutually exclusive vs. joint / mutually inclusive
    \item Subset $A \subset B$ with $\mathbb{P}(A) < \mathbb{P}(B)$
\end{itemize}

{\color{lightgray}\hrule height 0.001mm}

\emph{Probabilities} --- 
\begin{itemize}
    \item Marginal probability $\mathbb{P}(A)$: Probability for single variable: $p(\mathcal{X}) = \sum_{\mathcal{Y}}p(x,y)$ resp. $f(\mathcal{X}) = \int_{\mathcal{Y}}f(x,y)dy$
    \item Joint probability $\mathbb{P}(A \cap B)$: Probability for combination of variables, given by all possible combinations resp. convolution of their pdfs 
    \item Conditional probability $\mathbb{P}(A|B) = \frac{\mathbb{P}(A \cap B)}{\mathbb{P}(B)}$: Probability for variable, given other variable: $p(\mathcal{X}|\mathcal{Y}) = \frac{p(x,y)}{\sum_{\mathcal{X}}p(x,y)}$ resp. $f(\mathcal{X}|\mathcal{Y}) = \frac{f(x,y)}{\int_{\mathcal{X}}f(x,y)dy}$
    \begin{itemize}
        \item $\mathbb{P}(A|B) = 1 - \mathbb{P}(A^C|B)$
        \item $\mathbb{P}(A_1|B) + \mathbb{P}(A_2|B) + ... = 1$
        \item If conditioning on subset $S$:
        \begin{align*}
        p(x | S) = 
        \left\{
            \begin{aligned}
                 & p(x) / p(x \in S) \quad & x \in S \\ 
                 & 0 \quad & x \notin S  
            \end{aligned}
        \right.
        \end{align*}
    \end{itemize}
    \item Bayesian terminology: 
    \begin{itemize}
        \item Prior $\mathbb{P}(\textrm{parameter})$
        \item Posterior $\mathbb{P}(\textrm{parameter} | \textrm{data})$
        \item Likelihood $\mathbb{P}(\textrm{data} | \textrm{parameter})$
        \item Evidence $\mathbb{P}(\textrm{data})$
    \end{itemize}
    \item \emph{Bayes theorem}: $\textrm{Posterior } \mathbb{P}(A|B) = \frac{\textrm{Likelihood }\mathbb{P}(B|A) \times  \textrm{Prior }\mathbb{P}(A)}{\textrm{Evidence }\mathbb{P}(B)}$
    \item Attention! In $p(\cdot | \theta)$ the $|$ can either refer to parametrizing on $\theta$ (parameter is part of the distribution's form but isn't observed or fixed) or conditioning on $\theta$ (parameter takes a observed and fixed value, and we evaluate the distribution on this condition)
\end{itemize}

{\color{black}\hrule height 0.001mm}

\subsection*{Measures}
\emph{$n^{th}$ moment} --- 
$\mathbb{E}(\mathcal{X}^n) = \int_{-\infty}^{\infty}x^n \times f(x)dx$

{\color{lightgray}\hrule height 0.001mm}

\emph{Expected value} --- 
$\mathbb{E}(\mathcal{X}) = \sum_{\mathcal{X}}x \times p(x)$ resp. $\mathbb{E}(\mathcal{X}) = \int_{-\infty}^{\infty}x \times f(x)dx$ with pmf resp. pdf --- Properties:
\begin{multicols}{2}
\begin{itemize}
    \item $\mathbb{E}(\alpha)=\alpha$
    \item $\mathbb{E}(\alpha\mathcal{X}+\beta)=\alpha\mathbb{E}(\mathcal{X})+\beta$
    \item $\mathbb{E}(\alpha\mathcal{X} + \beta\mathcal{Y})=\alpha\mathbb{E}(\mathcal{X})+\beta\mathbb{E}(\mathcal{Y})$
    \item For orthogonal variables:
    \begin{itemize}
        \item  $\mathbb{E}(\mathcal{X}\mathcal{Y})=0$
        \item $\mathbb{E}((\mathcal{X}+\mathcal{Y})^2)=\mathbb{E}(\mathcal{X}^2) + \mathbb{E}(\mathcal{Y}^2)$
    \end{itemize}
    \item For independent variables: $\mathbb{E}(\mathcal{X}\mathcal{Y})=\mathbb{E}(\mathcal{X})\mathbb{E}(\mathcal{Y})$
    \item For vector $\boldsymbol{y} = \boldsymbol{A}\boldsymbol{x}$:\\
    $\mathbb{E}_y(\boldsymbol{A}\boldsymbol{x})=\boldsymbol{A}\mathbb{E}_X(\boldsymbol{x})$
\end{itemize}
\end{multicols}
\begin{itemize}
    \item For conditional probability ($A$ is an event, $X$ is a random variable): \\
    \begin{itemize}
        \item $\mathbb{E}[P(X | A)] = \sum_x P(X=x) \times P(X=x | A) $ resp. $\int_{-\infty}^{\infty} f_X(x) \times f_{X|A}(x)dx$ 
        \item $\mathbb{E}[P(A | X)] = P(A) = \sum_x P(X=x) \times P(A | X=x) $ resp. $\int_{-\infty}^{\infty} f_X(x) \times P(A|X=x) dx$ 
    \end{itemize}
    \item For conditional events resp. variables ($A$ is an event, $X$ is a random variable): \\
    \begin{itemize}
        \item $\mathbb{E}(X | A) = \sum_x x \times P(X=x | A) $ resp. $\int_{-\infty}^{\infty} x \times f_{X|A}(x)dx$ 
        \item $\mathbb{E}(A | X) = P(A|X)$
        \item $\mathbb{E}[\mathbb{E}(X | A)] = \mathbb{E}(X)$\\
        Proof:
        \begin{itemize}
            \item $\mathbb{E}[\mathbb{E}(X | A)] = \int_{-\infty}^{\infty} f_A(a) \mathbb{E}(X | A) da = \int_{-\infty}^{\infty} f_A(a) \int_{-\infty}^{\infty} x f_{X|A}(x|a) dx da = \int_{-\infty}^{\infty} \int_{-\infty}^{\infty} x f_{X,A}(x,a) dx da = \int_{-\infty}^{\infty} x \int_{-\infty}^{\infty} f_{X,A}(x,a) da dx = \int_{-\infty}^{\infty} x f_{X}(x) dx = \mathbb{E}(X)$
        \end{itemize}
    \end{itemize}
    \item \emph{Cauchy Schwarz inequality}: $\mathbb{E}(\mathcal{X},\mathcal{Y})^2 \leq \mathbb{E}(\mathcal{X}^2)\mathbb{E}(\mathcal{Y}^2)$
\end{itemize}

{\color{lightgray}\hrule height 0.001mm}

\emph{Median} --- Real number $M$ defined by $P(X<M) = P(X>M)$

{\color{lightgray}\hrule height 0.001mm}

\emph{Standard deviation} --- $\sqrt{\mathbb{V}(\mathcal{X})}$

{\color{lightgray}\hrule height 0.001mm}

\emph{Covariance} ---
\begin{itemize}
    \item Univariate variance of a random variable: $\mathbb{V}(\mathcal{X}) = \mathbb{E}((\mathcal{X}-\mathbb{E}(\mathcal{X}))^2) = \mathbb{E}(\mathcal{X}^2) - \mathbb{E}(\mathcal{X})^2$ where $\mathbb{E}(\mathcal{X}^2)$ is the unnormalized correlation resp. inner product
    \item Univariate covariance of two random variables: $\textrm{Cov}(\mathcal{X}, \mathcal{Y}) = \mathbb{E}((\mathcal{X}-\mathbb{E}(\mathcal{X}))(\mathcal{Y}-\mathbb{E}(\mathcal{Y}))) = \mathbb{E}(\mathcal{X}\mathcal{Y}) - \mu_{\mathcal{X}} \mu_{\mathcal{Y}}$ where $\mathbb{E}(\mathcal{X}\mathcal{Y})$ is the unnormalized correlation resp. inner product
    \item Proof (schematically for variance): $\mathbb{V}(\mathcal{X}) = \mathbb{E}((\mathcal{X}-\mathbb{E}(\mathcal{X}))^2) = \mathbb{E}[ \mathcal{X}^2 - \mathcal{X}\mathbb{E}(\mathcal{X}) - \mathcal{X}\mathbb{E}(\mathcal{X}) + \mathbb{E}(\mathcal{X})^2 ] = \mathbb{E}[ \mathcal{X}^2] - \mathbb{E}[\mathcal{X}]\mathbb{E}(\mathcal{X}) - \mathbb{E}[\mathcal{X}]\mathbb{E}(\mathcal{X}) + \mathbb{E}(\mathcal{X})^2 = \mathbb{E}(\mathcal{X}^2) - \mathbb{E}(\mathcal{X})^2$ where $\mathbb{E}(\mathcal{X}^2)$ 
    \item Multivariate covariance matrix of a vector: 
    \begin{itemize}
        \item $\boldsymbol{\Sigma} = \textrm{Cov}(\boldsymbol{\mathcal{X}}) = \mathbb{E}((\boldsymbol{\mathcal{X}}-\mathbb{E}(\boldsymbol{\mathcal{X}}))(\boldsymbol{\mathcal{X}}-\mathbb{E}(\boldsymbol{\mathcal{X}}))^\intercal) = \mathbb{E}(\boldsymbol{\mathcal{X}}\boldsymbol{\mathcal{X}}^\intercal) - \mathbb{E}(\boldsymbol{\mathcal{X}})\mathbb{E}(\boldsymbol{\mathcal{X}})^\intercal = \begin{bmatrix}
        \textrm{Var}(\mathcal{X}_1) & ... & \textrm{Cov}(\mathcal{X}_1,\mathcal{X}_m) \\
        ... & ... & ... \\
        \textrm{Cov}(\mathcal{X}_m,\mathcal{X}_1) & ... & \textrm{Var}(\mathcal{X}_m)
        \end{bmatrix}$ where $\boldsymbol{R} = \mathbb{E}(\boldsymbol{\mathcal{X}}\boldsymbol{\mathcal{X}}^\intercal)$ is the unnormalized correlation matrix
        \item $\boldsymbol{\Sigma}$ and $\boldsymbol{R}$ are symmetric and psd
        \item $\boldsymbol{\Sigma} = \boldsymbol{R} - \boldsymbol{\mu}_X \boldsymbol{\mu}_X^\intercal$
    \end{itemize}
\end{itemize}
Properties - variance:
\begin{multicols}{2}
\begin{itemize}
    \item $\mathbb{V}(\alpha)=0$
    \item $\mathbb{V}(\alpha\mathcal{X}+\beta)=\alpha^2\mathbb{V}(\mathcal{X})$
    \item $\mathbb{V}(\mathcal{X} + \mathcal{Y})=\mathbb{V}(\mathcal{X})+2\textrm{Cov}(\mathcal{X},\mathcal{Y})+\mathbb{V}(\mathcal{Y})$
    \item For uncorrelated (and independent) variables: $\mathbb{V}(\mathcal{X} + \mathcal{Y})=\mathbb{V}(\mathcal{X})+\mathbb{V}(\mathcal{Y})$
    \item For independent variables: $\mathbb{V}(\mathcal{X}\mathcal{Y})=\mathbb{E}((\mathcal{X}\mathcal{Y})^2)\mathbb{E}(\mathcal{X}\mathcal{Y})^2$
    \item For vector $\boldsymbol{y} = \boldsymbol{A}\boldsymbol{x}$:\\ $\mathbb{V}_y = \boldsymbol{A}\mathbb{V}_X\boldsymbol{A}^\intercal$
    \item For zero-mean variable: $\mathbb{V}(\mathcal{X})= \mathbb{E}(\mathcal{X}^2) - \mathbb{E}(\mathcal{X})^2 = \mathbb{E}(\mathcal{X}^2)$ since $\mathbb{E}(\mathcal{X}) = 0$
\end{itemize}
\end{multicols}
Properties - covariance:
\begin{itemize}
    \item $\textrm{Cov}(\mathcal{X},\mathcal{X}) = \mathbb{V}(\mathcal{X})$
    \item $\textrm{Cov}((\alpha \mathcal{X} + \beta \mathcal{Y}),\mathcal{Z}) = \alpha \textrm{Cov}(\mathcal{X},\mathcal{Z}) + \beta \textrm{Cov}(\mathcal{Y},\mathcal{Z})$
    \item If covariance of two random variables is 0, they are uncorrelated, but not necessarily independent. Then, $\mathbb{E}(\mathcal{X}\mathcal{Y})=\mathbb{E}(\mathcal{X})\mathbb{E}(\mathcal{Y})$
    \item If covariance and unnormalized correlation of two random variables is 0, they are orthogonal, but not necessarily independent. Then, $\mathbb{E}(\mathcal{X}\mathcal{Y}) = 0$
    \item For vector $\boldsymbol{y} = \boldsymbol{A}\boldsymbol{x}$:
        \begin{itemize}
            \item $\boldsymbol{\Sigma}_y = \boldsymbol{A}\boldsymbol{\Sigma}_X\boldsymbol{A}^\intercal$
            \item $\boldsymbol{R}_y = \boldsymbol{A}\boldsymbol{R}_X\boldsymbol{A}^\intercal$
        \end{itemize}
    \item For zero-mean variables: $\textrm{Cov}(\mathcal{X},\mathcal{Y})= \mathbb{E}(\mathcal{X}\mathcal{Y}) - \mu_{\mathcal{X}} \mu_{\mathcal{Y}} = \mathbb{E}(\mathcal{X},\mathcal{Y})$ since $\mu_{\mathcal{X}} = \mu_{\mathcal{Y}} = 0$
\end{itemize}
\emph{Cauchy Schwarz inequality}: \begin{itemize}
    \item $\textrm{Cov}(\mathcal{X},\mathcal{Y})^2 \leq \mathbb{V}(\mathcal{X})\mathbb{V}(\mathcal{Y})$
    \item $\mathbb{E}(\mathcal{X}\mathcal{Y})^2 \leq \mathbb{E}(\mathcal{X}^2)\mathbb{V}(\mathcal{Y}^2)$
\end{itemize}

{\color{lightgray}\hrule height 0.001mm}

\emph{Correlation} --- Normalized covariance
\begin{itemize}
    \item Univariate correlation of a random variable: $\textrm{Cor}(\mathcal{X}, \mathcal{Y}) = \frac{\textrm{Cov}(\mathcal{X}, \mathcal{Y})}{\sqrt{\mathbb{V}(\mathcal{X})} \sqrt{\mathbb{V}(\mathcal{Y})}}$ 
    \item Multivariate correlation matrix of a vector: 
    \begin{itemize}
        \item $\boldsymbol{P} = \textrm{Cor}(\boldsymbol{\mathcal{X}}) = \begin{bmatrix}
        1 & ... & \textrm{Cor}(\mathcal{X}_1,\mathcal{X}_m) \\
        ... & ... & ... \\
        \textrm{Cor}(\mathcal{X}_m,\mathcal{X}_1) & ... & 1
        \end{bmatrix}$
        \item $\boldsymbol{P}$ is symmetric and psd
        \item Correlation is bounded between 0 and 1, given Cauchy Schwarz Inequality
        \item If correlation of two random variables is 0, they are not necessarily independent
    \end{itemize}
\end{itemize}

{\color{lightgray}\hrule height 0.001mm}

\emph{Skewness} --- $3^{rd}$ moment \hl{TBA}

{\color{lightgray}\hrule height 0.001mm}

\emph{Kurtosis} --- $4^{th}$ moment \hl{TBA}

{\color{black}\hrule height 0.001mm}

\subsection*{Probability Distributions}
\emph{PMF, CDF, PDF} --- 
\begin{itemize}
    \item Cumulative density function $F(r)$ (CDF): $F(r) = p(x \leq r)$
    \item Probability mass function $p(x)$ (PMF) for discrete random variables: $p(x)$
    \item Probability density function (PDF) $f(x)$ for continuous random variables: $\int_{-\infty}^r f(x)dx = p(x \leq r) = \int_{-\infty}^{\infty} \int_{-\infty}^r f(x,y) dx dy = p(x \leq r) = F(r)$
    \item Properties of CDF and PDF:
    \begin{itemize}
        \item Derivative of CDF returns PDF, integral of PDF returns CDF
        \item Monotonically non-decreasing: If $s<r, F(s) < F(r)$
        \item $lim_{r\rightarrow-\infty} F(r) = 0$
        \item $lim_{r\rightarrow\infty} F(r) = 1$
        \item Right-continuous: $lim_{s\rightarrow-r^+} F(s) = F(r)$
        \item $lim_{s\rightarrow-r^-} F(s) = F(x < r) = F(s) - F(x = r)$
        \item $\int_a^b f(x)dx = F(b)-F(a) = p(a < x \leq b)$
        \item $\int_{-\infty}^\infty f(x)dx = 1$
    \end{itemize}
\end{itemize}

{\color{lightgray}\hrule height 0.001mm}

\emph{Normal distribution} --- 
$\mathcal{X} \sim \mathcal{N}(\mu, \sigma^2)$\\
For univariate, PDF: $\frac{1}{\sqrt{2\pi\sigma}} exp(\frac{-(x-\mu)^2}{2\sigma^2})$\\
For multivariate, PDF: $\frac{1}{{2\pi\sigma}^{n/2}} \frac{1}{|\boldsymbol{\Sigma}|^{1/2}} exp(-\frac{1}{2} (\boldsymbol{x}-\boldsymbol{\mu})^\intercal \boldsymbol{\Sigma}^{-1} (\boldsymbol{x}-\boldsymbol{\mu}))$ where the term in the exponent is a quadratic form\\
Convolution: $\int \mathcal{N}(a;Bc,D) \times \mathcal{N}(c;e,F) dc = \int \mathcal{N}(a;Be,D + BFB^\intercal$

{\color{lightgray}\hrule height 0.001mm}

\emph{Standard normal distribution} --- Normal distribution, standardized via z-score $z = \frac{x-\mu}{\sigma}$, which results in $\mu = 0$ and $\sigma = 1$

{\color{lightgray}\hrule height 0.001mm}

\emph{Bernoulli distribution} --- trial with success (probability $p$) or failure (probability $1-p$)
\begin{multicols}{2}
\begin{itemize}
    \item $\mathcal{X} \sim \textrm{Bernoulli}(p)$
    \item PDF: $p(x) p^x (1-p)^x$
    \item Mean: $\mathbb{E}(x) = p$
    \item Variance: $\mathbb{V}(x) = p(1-p)$
\end{itemize}
\end{multicols}

{\color{lightgray}\hrule height 0.001mm}

\emph{Binomial distribution} --- $n$ independent Bernoulli trials with $k$ successes
\begin{multicols}{2}
\begin{itemize}
    \item $\mathcal{X} \sim \textrm{Bin}(n,p)$
    \item PDF: $\binom{n}{k} p^k (1-p)^{n-k}$
    \item Mean: $\mathbb{E}(x) = np$
    \item Variance: $\mathbb{V}(x) = np(1-p)$
\end{itemize}
\end{multicols}

{\color{lightgray}\hrule height 0.001mm}

\emph{Poisson distribution} ---
\begin{multicols}{2}
\begin{itemize}
    \item $\mathcal{X} \sim \textrm{Pois}(\lambda)$
    \item PDF: $e^{-\lambda} \frac{\lambda^x}{x!}$
    \item Mean: $\mathbb{E}(x) = \lambda$
    \item Variance: $\mathbb{V}(x) = \lambda$
\end{itemize}
\end{multicols}

{\color{lightgray}\hrule height 0.001mm}

\emph{Beta distribution} ---
\begin{itemize}
    \item $X$ takes values $\in [0,1]$
    \item Represents the probability of a Bernoulli process after observing $\alpha-1$ successes and $\beta-1$ failures
\end{itemize}
\begin{multicols}{2}
\begin{itemize}
    \item $\mathcal{X} \sim \textrm{Beta}(\alpha,\beta)$ where $\alpha,\beta > 0$
    \item PDF: $\frac{\Gamma(\alpha+\beta)}{\Gamma(\alpha)+\Gamma(\beta)} x^{\alpha-1} (1-x)^{\beta-1}$
    where $\Gamma(\alpha) = \int_0^\infty u^{\alpha-1} e^{-u} du$
    \item Mean: $\mathbb{E}(x) = \frac{\alpha}{\alpha+\beta}$
    \item Variance: $\mathbb{V}(x) = \frac{\alpha\beta}{(\alpha+\beta)^2(\alpha+\beta+1)}$
\end{itemize}
\end{multicols}

{\color{lightgray}\hrule height 0.001mm}

\emph{Dirichlet distribution} ---
\begin{itemize}
    \item $X$ takes values $\in [0,1]$
    \item Multivariate extension of Beta distribution
    \item $Dir(\boldsymbol{x} | \boldsymbol{\alpha}) = \frac{1}{B(\boldsymbol{\alpha})} \prod_{k=1}^n u_k^{\alpha_k - 1}$,
    where $B(\boldsymbol{\alpha})$ is the multivariate generalization of the Beta function:
    $B(\boldsymbol{\alpha}) = \frac{\prod_{k=1}^n \Gamma(\alpha_k)}{\Gamma\left(\sum_{k=1}^n \alpha_k\right)}$
\end{itemize}

{\color{lightgray}\hrule height 0.001mm}

\emph{Uniform distribution} ---
\begin{multicols}{2}
\begin{itemize}
    \item Assume $x$ is uniformly distributed between $[a,b]$
    \item PDF: $f(x) =$\\
        $\left\{
            \begin{aligned}
                 & \frac{1}{b-a} \quad & a \leq x \leq b \\
                 & 0 & \textrm{ otherwise}
            \end{aligned}
        \right.$
    \item CDF: $F(x) =$\\
        $\left\{
            \begin{aligned}
                 & 0 & x < a \\
                 & \frac{x-a}{b-a} & a \leq x \leq b \\
                 & 1 & x > b
            \end{aligned}
        \right.$
\end{itemize}
\end{multicols}

{\color{lightgray}\hrule height 0.001mm}

\emph{Exponential families} ---  Family of probability distributions (incl. Gaussian, Poisson, Bernoulli, Categorical, Gamma, Beta) of the form: $ p(x \mid \boldsymbol{\theta}) = \frac{1}{Z(\boldsymbol{\theta})} \boldsymbol{h}(x) \exp\left(\boldsymbol{\theta} \cdot \boldsymbol{\Phi}(x)\right) $ where
\begin{itemize}
\item $Z(\boldsymbol{\theta})$ is a \emph{partition function} which acts as a normalization constant to $[0, 1]$
\item $\boldsymbol{h}(x)$ determines the \emph{support} of the function
\item $\boldsymbol{\theta}$ are the \emph{parameters}
\item $\boldsymbol{\Phi}(x)$ are the \emph{sufficient statistics}
\end{itemize}
Properties:
\begin{itemize}
    \item Finite sufficient statistics
    \item \emph{Conjugate priors}:
    \begin{itemize}
        \item Prior distribution $p$ for a likelihood distribution $q$ such that the posterior distribution induced by $p$ and $q$ is of the same form as $p$
        \item This helps to conduct Bayesian machine learning: Posterior parameters can be updated using prior parameters and do not have to be re-derived
        \item Example: Bernoulli distribution with Beta prior:
        \begin{itemize}
            \item Assume prior $P(\theta) \sim \textrm{Beta}(\alpha, \beta)$
            \item Likelihood $P(X \mid \theta) = \theta^k (1 - \theta)^{n-k}$, where $n = \textrm{trials}$ and $k = \textrm{successes}$
            \item Posterior $P(\theta \mid X) \propto P(X \mid \theta) P(\theta) \propto \theta^k (1 - \theta)^{n-k} \theta^{\alpha-1} (1 - \theta)^{\beta-1} \propto \theta^{k + \alpha - 1} (1 - \theta)^{n - k + \beta - 1}$
            \item Due to conjugate prior, posterior $\sim \textrm{Beta}(\alpha', \beta')$, with $\alpha' = k + \alpha$ and $\beta' = n - k + \beta$
        \end{itemize}
    \end{itemize}
    \item Corresponds to maximum entropy distributions, if sufficient statistics match training data (see below)
\end{itemize}
Derivation from maximum entropy:
\begin{itemize}
    \item Maximum entropy principle: When estimating a probability distribution over some constraints (e.g. known \mathbb{E}), we should choose the distribution that has the highest entropy
    \item Entropy for discrete $x$:
    $
    H(x) = -\sum_{x \in \mathcal{X}} p(x) \log p(x)
    $
    \item Assume we know $\mathbb{E}(\boldsymbol{\Phi}(x)) = F$
    \item Using the maximum entropy principle, we want to find $J(p) = \sum_{x \in \mathcal{X}} p(x) \log p(x)$ subject to
    \begin{itemize}
        \item Non-negativity constraint: $p(x) \geq 0 \quad \forall x$
        \item Sum-to-one constraint: $\sum_{x \in \mathcal{X}} p(x) = 1$
        \item User-defined constraint: $\mathbb{E}(\boldsymbol{\Phi}(x)) = \sum_{x \in \mathcal{X}} p(x) \boldsymbol{\Phi}(x) = F$
    \end{itemize}
    \item This is a constrained optimization problem with the objective function: $
    J(p, \lambda) = -\sum_{x \in \mathcal{X}} p(x) \log p(x) - \lambda_0 \left(1 - \sum_{x \in \mathcal{X}} p(x)\right) + \sum_k \lambda_k \left(F_k - \sum_{x \in \mathcal{X}} p(x) \Phi_k(x)\right)
    $
    \item Then, the gradient is given by:
    $
    \frac{\partial J(p, \lambda)}{\partial p(x)} = -1 - \log(p(x)) - \lambda_0 - \sum_k \lambda_k \Phi_k(x) = 0
    $
    \item $\Rightarrow \log(p(x)) = -1 - \lambda_0 - \sum_k \lambda_k \Phi_k(x) = \log(\frac{1}{Z} \exp\left(-\sum_k \lambda_k \Phi_k(x)\right))$ where $Z = \exp(1 + \lambda_0)$
    \item $\Rightarrow p(x) = \frac{1}{Z} \exp\left(-\sum_k \lambda_k \Phi_k(x)\right)$
    \item $\Rightarrow
    \sum_{x \in \mathcal{X}} p(x) = \sum_{x \in \mathcal{X}} \frac{1}{Z} \exp\left(-\sum_k \lambda_k \Phi_k(x)\right) = 1
    $ by sum-to-one constraint
    \item $\Rightarrow
    Z = \sum_{x \in \mathcal{X}} \exp\left(-\sum_k \lambda_k \Phi_k(x)\right)$
    \item If we plug this $Z$ back into $p(x)$, we get:
    $
    p(x) = \frac{\exp\left(-\sum_k \lambda_k \Phi_k(x)\right)}{\sum_{x \in \mathcal{X}} \exp\left(-\sum_k \lambda_k \Phi_k(x)\right)} = \frac{1}{\sum_{x \in \mathcal{X}} \exp\left(-\boldsymbol{\lambda}^\intercal \boldsymbol{\Phi}(x)\right)} \exp\left(-\boldsymbol{\lambda}^\intercal \boldsymbol{\Phi}(x)\right)
    $
    \item Then, we see that this is in the exponential family
\end{itemize}
Proof that Gaussian is an exponential distribution:
\begin{itemize}
    \item Gaussian PDF:
    $
    f(x \mid \mu, \sigma^2) = \frac{1}{\sqrt{2 \pi \sigma^2}} e^{-\frac{(x - \mu)^2}{2 \sigma^2}} $
    \item $= \frac{1}{\sqrt{2 \pi \sigma^2}} e^{-\frac{x^2 - 2 \mu x + \mu^2}{2 \sigma^2}}$
    \item $=\frac{1}{\sqrt{2 \pi \sigma^2}} e^{-\frac{x^2}{2 \sigma^2} + \frac{2 \mu x}{2 \sigma^2} - \frac{\mu^2}{2 \sigma^2}}$
    \item $= \frac{1}{\sqrt{2 \pi \sigma^2}} e^{-\frac{x^2}{2 \sigma^2}} e^{\frac{\mu x}{\sigma^2}} e^{-\frac{\mu^2}{2 \sigma^2}} 
    $
    \item $= \frac{e^{-\frac{\mu^2}{2 \sigma^2}}}{\sqrt{2 \pi \sigma^2}} \times 1 \times e^{\frac{\mu x}{\sigma^2}} e^{-\frac{x^2}{2 \sigma^2}}$
    \item $= \frac{e^{-\frac{\mu^2}{2 \sigma^2}}}{\sqrt{2 \pi \sigma^2}} \times 1 \times e^{\frac{\mu}{\sigma^2}x-\frac{1}{2 \sigma^2} x^2}$
    \item $= \frac{e^{-\frac{\mu^2}{2 \sigma^2}}}{\sqrt{2 \pi \sigma^2}} \times 1 \times e^{[\frac{\mu}{\sigma^2},\frac{-1}{2\sigma^2}] \cdot [x,x^2]}$
    \item Thus, parameters of exponential family given by:
    \begin{itemize}
        \item $\frac{1}{Z(\boldsymbol{\theta})} = \frac{e^{-\frac{\mu^2}{2 \sigma^2}}}{\sqrt{2 \pi \sigma^2}}$
        $ \Rightarrow
        Z(\boldsymbol{\theta}) = \sqrt{2 \pi \sigma^2} \times \frac{1}{e^{-\frac{\mu^2}{2 \sigma^2}}} = \sqrt{2 \pi \sigma^2} e^{\frac{\mu^2}{2 \sigma^2}}
        $
        \item $\boldsymbol{h}(x) = 1$
        \item $\boldsymbol{\theta} = \begin{bmatrix}
        \mu / \sigma^2 \\
        -1 / 2 \sigma^2
        \end{bmatrix}$
        \item $
        \boldsymbol{\Phi}(x) = \begin{bmatrix}
        x \\
        x^2
        \end{bmatrix}
        $
    \end{itemize}
\end{itemize}
Proof that Poisson is an exponential distribution:
\begin{itemize}
    \item Poisson PDF:
    $
    f(x \mid \lambda) = \frac{\lambda^x e^{-\lambda}}{x!}
    $
    \item $
    = \frac{1}{x!}e^{-\lambda} \lambda^x
    $
    \item $
    = \frac{1}{e^\lambda}\frac{1}{x!} e^{log(\lambda^x)}
    $
    \item Thus, parameters of exponential family given by:
    \begin{itemize}
        \item $\frac{1}{Z(\boldsymbol{\theta})} = \frac{1}{e^\lambda}$
        $ \Rightarrow
        Z(\boldsymbol{\theta}) = e^\lambda
        $
        \item $\boldsymbol{h}(x) = \frac{1}{x!}$
        \item $\boldsymbol{\theta} = \begin{bmatrix}
        log(\lambda)
        \end{bmatrix}$
        \item $
        \boldsymbol{\Phi}(x) = \begin{bmatrix}
        x
        \end{bmatrix}
        $
    \end{itemize}
\end{itemize}
Proof that Beta is an exponential distribution:
\begin{itemize}
    \item Beta PDF: $
    f(x \mid \alpha, \beta) = \frac{x^{\alpha - 1} (1 - x)^{\beta - 1}}{B(\alpha, \beta)}
    $
    \item 
    $
    = \frac{1}{B(\alpha, \beta)} \times 1 \times  e^{(\alpha - 1) \ln(x) + (\beta - 1) \ln(1 - x)}
    $
    \item 
    $
    = \frac{1}{B(\alpha, \beta)} \times 1 \times  e^{[(\alpha - 1),(\beta - 1)] \cdot [\ln(x), \ln(1 - x)]}
    $
    \item Thus, parameters of exponential family given by:
    \begin{itemize}
        \item $\frac{1}{Z(\alpha, \beta)} = \frac{1}{B(\alpha, \beta)}$
        $ \Rightarrow
        Z(\alpha, \beta) = B(\alpha, \beta)
        $
        $ \Rightarrow
        Z(\boldsymbol{\theta}) = B(\theta_1 + 1, \theta_2 + 1)
        $
        \item $\boldsymbol{h}(x) = 1$
        \item $\boldsymbol{\theta} = \begin{bmatrix}
        (\alpha - 1)\\
        (\beta - 1)
        \end{bmatrix}$
        \item $
        \boldsymbol{\Phi}(x) = \begin{bmatrix}
        \ln(x)\\
        \ln(1-x)
        \end{bmatrix}
        $
    \end{itemize}
\end{itemize}
Proof that Chi Square is an exponential distribution:
\begin{itemize}
    \item Chi Square PDF: $
    f(x \mid k) = \frac{x^{k/2 - 1} e^{-x/2}}{2^{k/2} \Gamma(k/2)}
    $
    \item 
    $
    = \frac{1}{2^{k/2}\Gamma(k/2)} e^{- x/2} e^{(k/2 - 1) \ln(x)}
    $
    \item Thus, parameters of exponential family given by:
    \begin{itemize}
        \item $\frac{1}{Z(k)} = \frac{1}{2^{k/2}\Gamma(k/2)}$
        $ \Rightarrow
        Z(k) = 2^{k/2}\Gamma(k/2)
        $
        $ \Rightarrow
        Z(\boldsymbol{\theta}) = 2^{\boldsymbol{\theta} +1}\Gamma(\boldsymbol{\theta} + 1)
        $
        \item $\boldsymbol{h}(x) = e^{- x/2}$
        \item $\boldsymbol{\theta} = \begin{bmatrix}
        k/2 - 1
        \end{bmatrix}$
        \item $
        \boldsymbol{\Phi}(x) = \begin{bmatrix}
        \ln(x)
        \end{bmatrix}
        $
    \end{itemize}
\end{itemize}
Proof that Binomial is an exponential distribution:
\begin{itemize}
    \item Binomial PDF: $
    f(x \mid n, \pi) = \binom{n}{x} \pi^x (1 - \pi)^{n - x}
    $
    \item 
    $
    = \binom{n}{x} e^{x \ln(\pi)}  e^{(n - x) \ln(1 - \pi)}
    $
    \item 
    $
    = \binom{n}{x} e^{x \ln(\pi) - x\ln(1 - \pi) + n\ln(1 - \pi)}
    $
    \item 
    $
    = \binom{n}{x} e^{x \ln(\frac{\pi}{1-\pi})} e^{n\ln(1 - \pi)}
    $
    \item 
    $
    = e^{n\ln(1 - \pi)} \binom{n}{x} e^{\ln(\frac{\pi}{1-\pi}) x}
    $
    \item Thus, parameters of exponential family given by:
    \begin{itemize}
        \item $\frac{1}{Z(\pi)} = e^{n\ln(1 - \pi)}$
        $ \Rightarrow
        Z(\pi) = \frac{1}{e^{n\ln(1 - \pi)}}
        $
        $ \Rightarrow
        Z(\boldsymbol{\theta}) = e^{n\ln(1 + e^{\boldsymbol{\theta}})}
        $
        \item $\boldsymbol{h}(x) = \binom{n}{x}$
        \item $\boldsymbol{\theta} = \begin{bmatrix}
        \ln(\frac{\pi}{1-\pi})
        \end{bmatrix}$
        \item $
        \boldsymbol{\Phi}(x) = \begin{bmatrix}
        x
        \end{bmatrix}
        $
    \end{itemize}
\end{itemize}

{\color{black}\hrule height 0.001mm}

\subsection*{Other Concepts}
\emph{Laws of large numbers} --- Sample mean of iid variables converges to population mean as $n \rightarrow \infty$

{\color{lightgray}\hrule height 0.001mm}

\emph{Jensen's inequality} --- Relates expected value of a convex function of a random variable to the convex function of the expected value of that random variable\\
$\mathbb{E}(f(\mathcal{X})) \geq f(\mathbb{E}(\mathcal{X}))$

{\color{lightgray}\hrule height 0.001mm}

\emph{Markov's inequality} --- $p(x \geq t) \leq \frac{\mathbb{E}(x)}{t}$\\
Interesting only for $t \geq \mathbb{E}(x)$ because $p(x \geq t)$ must then be less than or equal to 1\\
Generalizations:
\begin{itemize}
    \item $p(|x| \geq t) \leq \frac{\mathbb{E}(g(|x|))}{g(t)}$
    \item $p(|x| \geq t) \leq \frac{\mathbb{E}(|x|^n)}{t^n}$
\end{itemize}

{\color{lightgray}\hrule height 0.001mm}

\emph{Chebychev's inequality} --- $p( |x - \mu_x| \geq \alpha | \sigma_x |) \leq \frac{1}{\alpha^2}$\\
Interesting only for $\alpha > 1$\\
Implications:
\begin{itemize}
    \item For n variables: $p( |S_n - \mu_x| \geq \epsilon) \leq \frac{\sigma_x^2}{n \epsilon^2}$ where $S_n$ is the sample mean
\end{itemize}

{\color{lightgray}\hrule height 0.001mm}

\emph{Sufficient statistics} --- 
\begin{itemize}
    \item $Z=g(Y)$ is a sufficient statistic for estimating $X$ if $X$ can be estimated as well from $Z$ as from $Y$, i.e. condensing $Y$ to $Z$ does not entail any loss of information about $X$
    \item Conditioned on $Z$, $Y$ is independent of $X$: $p(Y|Z,X) = p(Y|Z)$
    \item For sufficient statistics, the MLE of $X$ from $Y$ is the same as the MLE of $X$ from $Z$: $argmag_x p(Y|X) \rho(y) = argmag_x p(Z|X) \rho(y)$
    \item $p(X|Z) = p(X|Y)$
    
\end{itemize}

{\color{black}\hrule height 0.001mm}

\subsection*{Hypothesis Testing}
\emph{Taxonomy} ---
\begin{itemize}
    \item Parametric tests:  Test-statistic follows a specific distribution, whose parameters are a function of the sample data
    \item Non-parametric tests: Test-statistic follows a specific distribution, whose parameters are not a function of the sample data, or we do not many any assumptions on which distribution the test statistic follows 
\end{itemize}

{\color{lightgray}\hrule height 0.001mm}

\emph{Terminology} ---
\begin{itemize}
    \item Hypothesis:
    \begin{itemize}
        \item $H_0$: Accepted null hypothesis, e.g. $p=p_0$, $p_1-p_2=p_{0,1}-p_{0,2}=0$
        \item $H_A$: Alternative hypothesis, e.g. $p \neq p_0$, $p_1-p_2 \neq p_{0,1}-p_{0,2} \neq 0$
    \end{itemize}
    \item Errors:
    \begin{itemize}
        \item True positive: Chose $H_0$, and $H_0$ obtains
        \item False negative, type I error: Chose $H_A$, but $H_0$ obtains
        \item True negative: Chose $H_A$, and $H_A$ obtains
        \item False positive, type II error: Chose $H_0$, but $H_A$ obtains
    \end{itemize}
    \item Significance level $\alpha$: 
    \begin{itemize}
        \item $\alpha \geq p(\textrm{type I error})$ with equality for continuous variables
        \item If $\alpha$ is small, the probability that we are erroneously rejecting $H_0$ is very small
        \item Set by us, typically at $5\%$
    \end{itemize}
    \item Critical value $z$: 
    \begin{itemize}
        \item For two-sided: $z_{\alpha/2}$, $z_{1 - \alpha/2}$
        \item For one-sided upper tail: $z_{1 - \alpha}$
        \item For one-sided lower tail: $z_{\alpha}$
        \item Associated z-score with $\alpha$
    \end{itemize}
    \item P-value $p$: 
    \begin{itemize}
        \item For two-sided: $p = P(|z| \geq z_n)$
        \item For one-sided upper tail: $p = P(z \geq z_n)$
        \item For one-sided lower tail: $p = P(z \leq z_n)$
        \item Probability, given $H_0$ that we observe a value as or more extreme as the observed value $z_n$ 
        \item Smallest significance level resp. largest confidence level, at which we can reject $H_0$ given the sample observed
        \item If p-value is less than significance level resp. if observed value is more extreme than critical value, reject $H_0$, because the probability that we are erroneously doing so is very small
    \end{itemize}
    \item Confidence level: $1-\alpha$, probability, given $H_0$, that we retain $H_0$
    \item Beta: $\beta = p(\textrm{type II error})$
    \item Power: $1-\beta$, probability, given $H_A$, that we reject $H_0$
    \item Test types:
    \begin{itemize}
        \item Two-sided: $H_0: p = p_0, H_A: p \neq p_0$
        \item One-sided upper tail: $H_0: p \leq p_0, H_A: p > p_0$
        \item One-sided lower tail: $H_0: p \geq p_0, H_A: p < p_0$
    \end{itemize}
\end{itemize}

{\color{lightgray}\hrule height 0.001mm}

\emph{Corrections for multiple test} ---
Bonferroni correction: New significance level set to $\alpha* = \alpha / K$
\hl{TBA}

{\color{lightgray}\hrule height 0.001mm}

\emph{Sample size calculation} --- \hl{TBA}

{\color{lightgray}\hrule height 0.001mm}

\emph{Neyman Pearson test} ---
\begin{itemize}
    \item Maximizes power while controlling type I errors
    \item Sets $\alpha$ such that $\alpha \geq p(\textrm{type I error})$

    \item Then minimizes $p(\textrm{type II error})$
    \item This is achieved by a likelihood-ratio test with threshold $\theta$, such that $\alpha$ equals or is as close as possible to $ p(\textrm{type I error})$: 
    \begin{itemize}
        \item If $\Lambda(x) = \frac{p(x|p_0)}{p(x|p_A)} > \theta$, we reject $H_0$
        \item Then, we have $P(\Lambda(x) > \theta | H_0) = P(\frac{p(x|p_0)}{p(x|p_A)} > \theta | H_0 ) = P(\textrm{type I error}) = \alpha$
        \item The smaller $\alpha$, the larger $\theta$
    \end{itemize}
\end{itemize}

{\color{lightgray}\hrule height 0.001mm}

\emph{Bayesian Hypothesis Testing} ---
\begin{itemize}
    \item If $\Lambda(x) = \frac{p(x|p_0)}{p(x|p_A)} > \theta$, we reject $H_0$
    \item $\theta = \frac{k(p_A,p_0)P(p_0)}{k(p_0,p_A)P(p_A)}$
    \item In this case, $\theta$ subsumes both the prior $p(x)$ and the costs $k(\hat{x},x)$
\end{itemize}

{\color{lightgray}\hrule height 0.001mm}

\emph{Parametric hypothesis tests (Z,T,F)} --- 
5x2 cross-validation paired t-test:
\begin{itemize}
    \item Compares the performance of two classifiers
    \item Split data in $2$ folds, train each classifier on one fold and test on the other, then swap the folds and repeat
    \item Repeat this process for $5$ rounds
    \item For both of the $2$ folds, record the performance difference between the two classifiers: $d_i = p_{1,i} - p_{2,i}$
    \item The mean performance for each of the $1,...,5$ rounds is $\bar{d} = \frac{d_1+d_2}{2}$, the variance is $s^2 = \frac{(d_1-\bar{d})^2 + (d_2-\bar{d})^2}{2}$
    \item Test statistic: $t = -\frac{\sum_{i=1}^5 \bar{d}_i}{\sqrt{\sum_{i=1}^5 s_i^2}}$
    \item Test statistic is compared to the critical value from the t-distribution with 5 degrees of freedom
    \item If p-value $< \alpha$, reject $H_0$
\end{itemize}
\hl{TBA}

{\color{lightgray}\hrule height 0.001mm}

\emph{Non-parametric hypothesis tests} --- 
McNemar’s test:
\begin{itemize}
    \item Compares the disagreement probability of two classifiers 
    \item In $2 \times 2$ contingency table (correct vs. incorrect predictions for each classifier) focuses on the off-diagonal elements $b$ and $c$ because these represent the cases where classifiers differ in outcomes, with one being correct, and the other incorrect
    \item $H_0: p_b = p_c$\\
    $H_A: p_b \neq p_c$
    \item Test statistic: $\mathcal{X}^2 = \frac{(b-c)^2}{b+c}$
    \item Test statistic is compared to the critical value from the chi-square distribution table with $1$ degree of freedom
    \item If p-value $< \alpha$, reject $H_0$
\end{itemize}
Permutation test:
\begin{itemize}
    \item Tests whether a classifier's performance is significantly better than random chance
    \item $H_0:$ Performance is random, there is no true relationship between the features and the response\\
    $H_A:$ Performance is not random
    \item Train classifier and record performance, then shuffle response labels while keeping the features unchanged (simulate $H_0$), re-train classifier and record performance, repeat
    \item Calculate the p-value, i.e. proportion of shuffled performance values that are greater than or equal to the initially observed performance
    \item If p-value $< \alpha$, reject $H_0$
\end{itemize}
\hl{TBA}

{\color{lightgray}\hrule height 0.001mm}

\emph{Confidence intervals} --- \hl{TBA}