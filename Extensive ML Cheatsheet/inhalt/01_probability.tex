\section{Probability and Statistics}
\subsection*{Terminology}
\emph{Kolmogorov axioms} --- 
Probability space defined by:
\begin{itemize}
    \item Sample space: All possible outcomes $\Omega = \{\omega_1,...,\omega_n\}$
    \item Event space: All possible results, where an event is a subset of the sample space 
    \item Probability measure: Function that assigns a probability to an event 
\end{itemize}
Axioms:
\begin{itemize}
    \item Event space must be a \emph{sigma algebra}:
    \begin{itemize}
        \item If $A$ is in sample space, its complement is also in sample space
        \item If $A_1,...A_n$ are in sample space, their union is also in sample space
    \end{itemize}
    \item Probability measure must satisfy:
    \begin{itemize}
        \item $0 \leq \mathbb{P}(A) \leq 1$
        \item $\mathbb{P}({\Omega}) = 1$
        \item If $A_1,A_2,...$ are in sample space and do not intersect, then $\mathbb{P}(A_1 \cup A_2 \cup ...) = \int_{n=1}^{\infty} \mathbb{P}(A_n)$
    \end{itemize}
\end{itemize}
Further properties:
\begin{itemize}
    \item All sets than can be formed from left and right inclusive interval $[0,a]$ are events. On that basis: $(b,1] = [0,b]^c \in$ event space.
\end{itemize}

{\color{lightgray}\hrule height 0.001mm}

\emph{Variables} --- 
\begin{itemize}
    \item Random variable:
    \begin{itemize}
        \item Discrete random variable: Characterized by pmf
        \item Continuous random variable: Characterized by pdf
    \end{itemize}
    \item Independent random variables:
    \begin{itemize}
        \item $\mathbb{P}(A|B) = \mathbb{P}(A)$ and $\mathbb{P}(B|A) = \mathbb{P}(B)$
        \item $\mathbb{E}(AB) = \mathbb{E}(A)\mathbb{E}(B)$
        \item Correlation is 0 
        \item $\mathbb{P}(A \cap B) = \mathbb{P}(A)\mathbb{P}(B)$
        \item Functions of independent random variables are also independent
    \end{itemize}
    \item Conditionally independent random variables: Two random variables $\mathcal{X}$ and $\mathcal{Y}$ are conditionally independent, if there is a confounder $\mathcal{L}$ that causally affects both variables, but if we control for this confounder, the variables are not causally connected
    \item I.I.D. random variables: Independent and from identical distribution
\end{itemize}

{\color{lightgray}\hrule height 0.001mm}

\emph{Events} --- 
\begin{itemize}
    \item Complement: $\mathbb{P}(A^C) = 1 - \mathbb{P}(A)$ and $\mathbb{P}(A \cup A^C) = \mathbb{P}(A)\mathbb{P}(A^C)$
    \item Disjoint / mutually exclusive vs. joint / mutually inclusive
    \item Subset $A \subset B$ with $\mathbb{P}(A) < \mathbb{P}(B)$
\end{itemize}

{\color{lightgray}\hrule height 0.001mm}

\emph{Probabilities} --- 
\begin{itemize}
    \item Marginal probability $\mathbb{P}(A)$: Probability for single variable: $p(\mathcal{X}) = \sum_{\mathcal{Y}}p(x,y)$ resp. $f(\mathcal{X}) = \int_{\mathcal{Y}}f(x,y)dy$
    \item Joint probability $\mathbb{P}(A \cap B)$: Probability for combination of variables, given by all possible combinations resp. convolution of their pdfs 
    \item Conditional probability $\mathbb{P}(A|B) = \frac{\mathbb{P}(A \cap B)}{\mathbb{P}(B)}$: Probability for variable, given other variable: $p(\mathcal{X}|\mathcal{Y}) = \frac{p(x,y)}{\sum_{\mathcal{X}}p(x,y)}$ resp. $f(\mathcal{X}|\mathcal{Y}) = \frac{f(x,y)}{\int_{\mathcal{X}}f(x,y)dy}$
    \begin{itemize}
        \item $\mathbb{P}(A|B) = 1 - \mathbb{P}(A^C|B)$
        \item $\mathbb{P}(A_1|B) + \mathbb{P}(A_2|B) + ... = 1$
        \item If conditioning on subset $S$:
        \begin{align*}
        p(x | S) = 
        \left\{
            \begin{aligned}
                 & p(x) / p(x \in S) \quad & x \in S \\ 
                 & 0 \quad & x \notin S  
            \end{aligned}
        \right.
        \end{align*}
    \end{itemize}
    \item Bayesian terminology: 
    \begin{itemize}
        \item Prior $\mathbb{P}(\textrm{parameter})$
        \item Posterior $\mathbb{P}(\textrm{parameter} | \textrm{data})$
        \item Likelihood $\mathbb{P}(\textrm{data} | \textrm{parameter})$
        \item Evidence $\mathbb{P}(\textrm{data})$
    \end{itemize}
    \item \emph{Bayes theorem}: $\textrm{Posterior } \mathbb{P}(A|B) = \frac{\textrm{Likelihood }\mathbb{P}(B|A) \times  \textrm{Prior }\mathbb{P}(A)}{\textrm{Evidence }\mathbb{P}(B)}$
    \item Attention! In $p(\cdot | \theta)$ the $|$ can either refer to parametrizing on $\theta$ (parameter is part of the distribution's form but isn't observed or fixed) or conditioning on $\theta$ (parameter takes a observed and fixed value, and we evaluate the distribution on this condition)
\end{itemize}

{\color{black}\hrule height 0.001mm}

\subsection*{Measures}
\emph{Expected value} --- 
$\mathbb{E}(\mathcal{X}) = \sum_{\mathcal{X}}x \times p(x)$ resp. $\mathbb{E}(\mathcal{X}) = \int_{-\infty}^{\infty}x \times f(x)dx$ with pmf resp. pdf --- Properties:
\begin{multicols}{2}
\begin{itemize}
    \item $\mathbb{E}(\alpha)=\alpha$
    \item $\mathbb{E}(\alpha\mathcal{X}+\beta)=\alpha\mathbb{E}(\mathcal{X})+\beta$
    \item $\mathbb{E}(\alpha\mathcal{X} + \beta\mathcal{Y})=\alpha\mathbb{E}(\mathcal{X})+\beta\mathbb{E}(\mathcal{Y})$
    \item For orthogonal variables:
    $\mathbb{E}((\mathcal{X}+\mathcal{Y})^2)=\mathbb{E}(\mathcal{X}^2) + \mathbb{E}(\mathcal{Y}^2)$
    \item For independent variables: $\mathbb{E}(\mathcal{X}\mathcal{Y})=\mathbb{E}(\mathcal{X})\mathbb{E}(\mathcal{Y})$
    \item For vector $\boldsymbol{y} = \boldsymbol{A}\boldsymbol{x}$:\\$\mathbb{E}_y(\boldsymbol{A}\boldsymbol{x})=\boldsymbol{A}\mathbb{E}_X(\boldsymbol{x})$
\end{itemize}
\end{multicols}
\emph{Cauchy Schwarz inequality}: $\mathbb{E}(\mathcal{X},\mathcal{Y})^2 \leq \mathbb{E}(\mathcal{X}^2)\mathbb{E}(\mathcal{Y}^2)$

{\color{lightgray}\hrule height 0.001mm}

\emph{Standard deviation} --- $\sqrt{\mathbb{V}(\mathcal{X})}$

{\color{lightgray}\hrule height 0.001mm}

\emph{Covariance} ---
\begin{itemize}
    \item Univariate variance of a random variable: $\mathbb{V}(\mathcal{X}) = \mathbb{E}((\mathcal{X}-\mathbb{E}(\mathcal{X}))^2) = \mathbb{E}(\mathcal{X}^2) - \mathbb{E}(\mathcal{X})^2$ where $\mathbb{E}(\mathcal{X}^2)$ is the unnormalized correlation resp. inner product
    \item Univariate covariance of two random variables: $\textrm{Cov}(\mathcal{X}, \mathcal{Y}) = \mathbb{E}((\mathcal{X}-\mathbb{E}(\mathcal{X}))(\mathcal{Y}-\mathbb{E}(\mathcal{Y}))) = \mathbb{E}(\mathcal{X}\mathcal{Y}) - \mu_{\mathcal{X}} \mu_{\mathcal{Y}}$ where $\mathbb{E}(\mathcal{X}\mathcal{Y})$ is the unnormalized correlation resp. inner product
    \item Multivariate covariance matrix of a vector: 
    \begin{itemize}
        \item $\boldsymbol{\Sigma} = \textrm{Cov}(\boldsymbol{\mathcal{X}}) = \mathbb{E}((\boldsymbol{\mathcal{X}}-\mathbb{E}(\boldsymbol{\mathcal{X}}))(\boldsymbol{\mathcal{X}}-\mathbb{E}(\boldsymbol{\mathcal{X}}))^\intercal) = \mathbb{E}(\boldsymbol{\mathcal{X}}\boldsymbol{\mathcal{X}}^\intercal) - \mathbb{E}(\boldsymbol{\mathcal{X}})\mathbb{E}(\boldsymbol{\mathcal{X}})^\intercal = \begin{bmatrix}
        \textrm{Var}(\mathcal{X}_1) & ... & \textrm{Cov}(\mathcal{X}_1,\mathcal{X}_m) \\
        ... & ... & ... \\
        \textrm{Cov}(\mathcal{X}_m,\mathcal{X}_1) & ... & \textrm{Var}(\mathcal{X}_m)
        \end{bmatrix}$ where $\boldsymbol{R} = \mathbb{E}(\boldsymbol{\mathcal{X}}\boldsymbol{\mathcal{X}}^\intercal)$ is the unnormalized correlation matrix
        \item $\boldsymbol{\Sigma}$ and $\boldsymbol{R}$ are symmetric and psd
        \item $\boldsymbol{\Sigma} = \boldsymbol{R} - \boldsymbol{\mu}_X \boldsymbol{\mu}_X^\intercal$
    \end{itemize}
\end{itemize}
Properties - variance:
\begin{multicols}{2}
\begin{itemize}
    \item $\mathbb{V}(\alpha)=0$
    \item $\mathbb{V}(\alpha\mathcal{X}+\beta)=\alpha^2\mathbb{V}(\mathcal{X})$
    \item $\mathbb{V}(\mathcal{X} + \mathcal{Y})=\mathbb{V}(\mathcal{X})+2\textrm{Cov}(\mathcal{X},\mathcal{Y})+\mathbb{V}(\mathcal{Y})$
    \item For uncorrelated (and independent) variables: $\mathbb{V}(\mathcal{X} + \mathcal{Y})=\mathbb{V}(\mathcal{X})+\mathbb{V}(\mathcal{Y})$
    \item For independent variables: $\mathbb{V}(\mathcal{X}\mathcal{Y})=\mathbb{E}((\mathcal{X}\mathcal{Y})^2)\mathbb{E}(\mathcal{X}\mathcal{Y})^2$
    \item For vector $\boldsymbol{y} = \boldsymbol{A}\boldsymbol{x}$:\\ $\mathbb{V}_y = \boldsymbol{A}\mathbb{V}_X\boldsymbol{A}^\intercal$
    \item For zero-mean variable: $\mathbb{V}(\mathcal{X})= \mathbb{E}(\mathcal{X}^2) - \mathbb{E}(\mathcal{X})^2 = \mathbb{E}(\mathcal{X}^2)$ since $\mathbb{E}(\mathcal{X}) = 0$
\end{itemize}
\end{multicols}
Properties - covariance:
\begin{itemize}
    \item $\textrm{Cov}(\mathcal{X},\mathcal{X}) = \mathbb{V}(\mathcal{X})$
    \item $\textrm{Cov}((\alpha \mathcal{X} + \beta \mathcal{Y}),\mathcal{Z}) = \alpha \textrm{Cov}(\mathcal{X},\mathcal{Z}) + \beta \textrm{Cov}(\mathcal{Y},\mathcal{Z})$
    \item If covariance of two random variables is 0, they are uncorrelated, but not necessarily independent. Then, $\mathbb{E}(\mathcal{X}\mathcal{Y})=\mathbb{E}(\mathcal{X})\mathbb{E}(\mathcal{Y})$
    \item If covariance and unnormalized correlation of two random variables is 0, they are orthogonal, but not necessarily independent. Then, $\mathbb{E}(\mathcal{X}\mathcal{Y}) = 0$
    \item For vector $\boldsymbol{y} = \boldsymbol{A}\boldsymbol{x}$:
        \begin{itemize}
            \item $\boldsymbol{\Sigma}_y = \boldsymbol{A}\boldsymbol{\Sigma}_X\boldsymbol{A}^\intercal$
            \item $\boldsymbol{R}_y = \boldsymbol{A}\boldsymbol{R}_X\boldsymbol{A}^\intercal$
        \end{itemize}
    \item For zero-mean variables: $\textrm{Cov}(\mathcal{X},\mathcal{Y})= \mathbb{E}(\mathcal{X}\mathcal{Y}) - \mu_{\mathcal{X}} \mu_{\mathcal{Y}} = \mathbb{E}(\mathcal{X},\mathcal{Y})$ since $\mu_{\mathcal{X}} = \mu_{\mathcal{Y}} = 0$
\end{itemize}
\emph{Cauchy Schwarz inequality}: \begin{itemize}
    \item $\textrm{Cov}(\mathcal{X},\mathcal{Y})^2 \leq \mathbb{V}(\mathcal{X})\mathbb{V}(\mathcal{Y})$
    \item $\mathbb{E}(\mathcal{X}\mathcal{Y})^2 \leq \mathbb{E}(\mathcal{X}^2)\mathbb{V}(\mathcal{Y}^2)$
\end{itemize}

{\color{lightgray}\hrule height 0.001mm}

\emph{Correlation} --- Normalized covariance
\begin{itemize}
    \item Univariate correlation of a random variable: $\textrm{Cor}(\mathcal{X}, \mathcal{Y}) = \frac{\textrm{Cov}(\mathcal{X}, \mathcal{Y})}{\sqrt{\mathbb{V}(\mathcal{X})} \sqrt{\mathbb{V}(\mathcal{Y})}}$ 
    \item Multivariate correlation matrix of a vector: 
    \begin{itemize}
        \item $\boldsymbol{P} = \textrm{Cor}(\boldsymbol{\mathcal{X}}) = \begin{bmatrix}
        1 & ... & \textrm{Cor}(\mathcal{X}_1,\mathcal{X}_m) \\
        ... & ... & ... \\
        \textrm{Cor}(\mathcal{X}_m,\mathcal{X}_1) & ... & 1
        \end{bmatrix}$
        \item $\boldsymbol{P}$ is symmetric and psd
        \item Correlation is bounded between 0 and 1, given Cauchy Schwarz Inequality
        \item If correlation of two random variables is 0, they are not necessarily independent
    \end{itemize}
\end{itemize}

{\color{black}\hrule height 0.001mm}

\subsection*{Probability Distributions}
\emph{PMF, CDF, PDF} --- 
\begin{itemize}
    \item Cumulative density function $F(r)$ (CDF): $F(r) = p(x \leq r)$
    \item Probability mass function $p(x)$ (PMF) for discrete random variables: $p(x)$
    \item Probability density function (PDF) $f(x)$ for continuous random variables: $\int_{-\infty}^r f(x)dx = p(x \leq r) = F(r)$
    \item Properties of CDF and PDF:
    \begin{itemize}
        \item Derivative of CDF returns PDF, integral of PDF returns CDF
        \item Monotonically non-decreasing: If $s<r, F(s) < F(r)$
        \item $lim_{r\rightarrow-\infty} F(r) = 0$
        \item $lim_{r\rightarrow\infty} F(r) = 1$
        \item Right-continuous: $lim_{s\rightarrow-r^+} F(s) = F(r)$
        \item $lim_{s\rightarrow-r^-} F(s) = F(x < r) = F(s) - F(x = r)$
        \item $\int_a^b f(x)dx = F(b)-F(a) = p(a < x \leq b)$
        \item $\int_{-\infty}^\infty f(x)dx = 1$
    \end{itemize}
\end{itemize}

{\color{lightgray}\hrule height 0.001mm}

\emph{Normal distribution} --- 
$\mathcal{X} \sim \mathcal{N}(\mu, \sigma^2)$\\
For univariate, PDF: $\frac{1}{\sqrt{2\pi\sigma}} exp(\frac{-(x-\mu)^2}{2\sigma^2})$\\
For multivariate, PDF: $\frac{1}{{2\pi\sigma}^{n/2}} \frac{1}{|\boldsymbol{\Sigma}|^{1/2}} exp(-\frac{1}{2} (\boldsymbol{x}-\boldsymbol{\mu})^\intercal \boldsymbol{\Sigma}^{-1} (\boldsymbol{x}-\boldsymbol{\mu}))$ where the term in the exponent is a quadratic form\\
Convolution: $\int \mathcal{N}(a;Bc,D) \times \mathcal{N}(c;e,F) dc = \int \mathcal{N}(a;Be,D + BFB^\intercal$

{\color{lightgray}\hrule height 0.001mm}

\emph{Bernoulli distribution} --- trial with success (probability $p$) or failure (probability $1-p$)
\begin{multicols}{2}
\begin{itemize}
    \item $\mathcal{X} \sim \textrm{Bernoulli}(p)$
    \item PDF: $p(x) p^x (1-p)^x$
    \item Mean: $\mathbb{E}(x) = p$
    \item Variance: $\mathbb{V}(x) = p(1-p)$
\end{itemize}
\end{multicols}

{\color{lightgray}\hrule height 0.001mm}

\emph{Binomial distribution} --- $n$ independent Bernoulli trials with $k$ successes
\begin{multicols}{2}
\begin{itemize}
    \item $\mathcal{X} \sim \textrm{Bin}(n,p)$
    \item PDF: $\binom{n}{k} p^k (1-p)^{n-k}$
    \item Mean: $\mathbb{E}(x) = np$
    \item Variance: $\mathbb{V}(x) = np(1-p)$
\end{itemize}
\end{multicols}

{\color{lightgray}\hrule height 0.001mm}

\emph{Poisson distribution} ---
\begin{multicols}{2}
\begin{itemize}
    \item $\mathcal{X} \sim \textrm{Pois}(\lambda)$
    \item PDF: $e^{-\lambda} \frac{\lambda^x}{x!}$
    \item Mean: $\mathbb{E}(x) = \lambda$
    \item Variance: $\mathbb{V}(x) = \lambda$
\end{itemize}
\end{multicols}

{\color{lightgray}\hrule height 0.001mm}

\emph{Beta distribution} ---
\begin{multicols}{2}
\begin{itemize}
    \item $\mathcal{X} \sim \textrm{Beta}(\alpha,\beta)$
    \item PDF: $\frac{\Gamma(\alpha+\beta)}{\Gamma(\alpha)+\Gamma(\beta)} x^{\alpha-1} (1-x)^{\beta-1}$
    \item Mean: $\mathbb{E}(x) = \frac{\alpha}{\alpha+\beta}$
    \item Variance: $\mathbb{V}(x) = \frac{\alpha\beta}{(\alpha+\beta)^2(\alpha+\beta+1)}$
\end{itemize}
\end{multicols}

{\color{black}\hrule height 0.001mm}

\subsection*{Laws of Large Numbers and Inqeualities}
\emph{Laws of large numbers} --- Sample mean of iid variables converges to population mean as $n \rightarrow \infty$

{\color{lightgray}\hrule height 0.001mm}

\emph{Jensen's inequality} --- Relates expected value of a convex function of a random variable to the convex function of the expected value of that random variable\\
$\mathbb{E}(f(\mathcal{X})) \geq f(\mathbb{E}(\mathcal{X}))$

{\color{lightgray}\hrule height 0.001mm}

\emph{Markov's inequality} --- $p(x \geq t) \leq \frac{\mathbb{E}(x)}{t}$\\
Interesting only for $t \geq \mathbb{E}(x)$ because $p(x \geq t)$ must then be less than or equal to 1\\
Generalizations:
\begin{itemize}
    \item $p(|x| \geq t) \leq \frac{\mathbb{E}(g(|x|))}{g(t)}$
    \item $p(|x| \geq t) \leq \frac{\mathbb{E}(|x|^n)}{t^n}$
\end{itemize}

{\color{lightgray}\hrule height 0.001mm}

\emph{Chebychev's inequality} --- $p( |x - \mu_x| \geq \alpha | \sigma_x |) \leq \frac{1}{\alpha^2}$\\
Interesting only for $\alpha > 1$\\
Implications:
\begin{itemize}
    \item For n variables: $p( |S_n - \mu_x| \geq \epsilon) \leq \frac{\sigma_x^2}{n \epsilon^2}$ where $S_n$ is the sample mean
\end{itemize}