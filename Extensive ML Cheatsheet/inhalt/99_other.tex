\section{Other}
\subsection*{Causal Models}
\emph{Causal scenarios} --- 
\begin{itemize}
    \item Causal scenario without selection bias: $\mathcal{X}$ affects $\mathcal{Y}$ and there is no selection bias 
    \begin{itemize}
        \item Some features $\mathcal{X}_{\bot Y}$ do not causally affect $\mathcal{Y}$, but are affected by $\mathcal{W}$
        \item Some features $\mathcal{X}_{\bot W}$ causally affect $\mathcal{Y}$, but are not affected by $\mathcal{W}$
        \item Some features $\mathcal{X}_{W \& Y}$ causally affect $\mathcal{Y}$ and are affected by $\mathcal{W}$ as well as $\mathcal{X}_{\bot Y}$ and $\mathcal{X}_{\bot W}$
    \end{itemize}
    \item Anti causal scenario: We assume $\mathcal{Y}$ affects $\mathcal{X}$, rather than the other way around
    \item Causal scenario with selection bias: $\mathcal{X}$ affects $\mathcal{Y}$ and there is a selection bias 
\end{itemize}

{\color{lightgray}\hrule height 0.001mm}

\emph{Counterfactual invariance} --- 
\begin{itemize}
    \item Counterfactual invariance: Results of estimator remain consistent across different counterfactual scenarios, i.e. if $\mathcal{Y}$ is affected by $\mathcal{X}$, and $\mathcal{X}$ is affected by $\mathcal{W}$, but $\mathcal{W}$ does not affect $\mathcal{Y}$, our estimator should be invariant to states of $\mathcal{W}$, i.e. $f(\mathcal{X}(\mathcal{W}_1)) = f(\mathcal{X}(\mathcal{W}_2))$
    \item For counterfactual invariance, the following must hold:
    \begin{itemize}
        \item Causal scenario without selection bias: $f(\mathcal{X}) \bot \mathcal{W}$, i.e. estimate $f$ only depends on $\mathcal{X}_{\bot W}$
        \item Anti causal scenario: $(f(\mathcal{X}) \bot \mathcal{W}) | \mathcal{Y}$, i.e. estimate $f$ only depends on $\mathcal{X}_{\bot W}$, provided $\mathcal{Y}$ is known
        \item Causal scenario with selection bias: $(f(\mathcal{X}) \bot \mathcal{W}) | \mathcal{Y}$ as long as $\mathcal{X}_{\bot Y}$ and $\mathcal{X}_{W \& Y}$ do not influence $\mathcal{Y}$ whatsoever, i.e. $(\mathcal{Y} \bot \mathcal{X}) | \mathcal{X}_{\bot W}, \mathcal{W}$
    \end{itemize}
    \item For causal scenario without selection bias we need to show: $\mathcal{X}_{\bot W} \bot \mathcal{W}$
    \item For anti causal scenario we need to show: $(\mathcal{X}_{\bot W} \bot \mathcal{W}) | \mathcal{Y}$
    \item This can be shown via \emph{d-separation}
\end{itemize}

{\color{lightgray}\hrule height 0.001mm}

\emph{D separation} --- 
\begin{itemize}
    \item Undirected path of $n$ nodes is d-separated, if it contains 3 nodes following any of the following forms and if this form is blocked: 
    \begin{itemize}
        \item Chain structure: $X \rightarrow Z \rightarrow Y$ or $Y \rightarrow Z \rightarrow X$ – is blocked, if we condition on $Z$, i.e. $Z$ is known
        \item Fork structure: $X \leftarrow Z \rightarrow Y$ – is blocked, if we condition on $Z$, i.e. $Z$ is known
        \item Collider structure: $X \rightarrow Z \leftarrow Y$ – is blocked, if we don't condition on $Z$ or any of its descendants
    \end{itemize}
    \item Random variables $X$ and $Y$ are conditionally independent if each path between them is d-separated\\
    $\rightarrow$ as soon as we have one blocked triple on path, entire path is blocked\\
    $\rightarrow$ as soon as one path is active, we cannot guarantee conditional independence
    \item For causal scenario without selection bias we can show $\mathcal{X}_{\bot W} \bot \mathcal{W}$ since all paths are blocked 
    \item For anti causal scenario we can show $(\mathcal{X}_{\bot W} \bot \mathcal{W}) | \mathcal{Y}$ since all paths are blocked, conditioned on $\mathcal{Y}$, i.e. if $\mathcal{Y}$ is observed
\end{itemize}

{\color{black}\hrule height 0.001mm}

\subsection*{Proofs}

\emph{Proofs} --- 
\begin{itemize}
    \item To prove $p \rightarrow q$:
    \begin{itemize}
        \item Prove $p \rightarrow \neg q$ is impossible
        \item Prove $\neg q \rightarrow \neg p$
    \end{itemize}
    \item To prove $p \leftrightarrow q$: Prove $p \rightarrow q$ and $q \rightarrow p$
    \item To prove statement by induction: 
    \begin{itemize}
        \item Prove base case for $n=0$ or $n=1$
        \item Assume inductive hypothesis: Assume statement holds for $n=k$
        \item Prove inductive step: Prove statement holds for $n=k+1$
    \end{itemize}
\end{itemize}